%**********************************************************************%
%*                                                                    *%
%*                 Licensed Materials - Property of IBM               *%
%*          � Copyright IBM Corp. 2000   All Rights Reserved          *%
%*                                                                    *%
%**********************************************************************%

\documentclass[12pt]{article} 

\usepackage{latex8}
\usepackage{times}
\usepackage{epsfig}

\renewcommand{\baselinestretch}{1.1}
\renewcommand\textfraction{0.0}
\renewcommand\topfraction{1.0}

\begin{document}

\ 
\vspace*{1.0in}

\begin{center}
\begin{LARGE}
{\bf DAISY Dynamic Binary Translation Software}\\
\end{LARGE}
\vspace*{0.5in}
{\bf Erik R. Altman}\\
{\bf Kemal Ebcio\u{g}lu}\\
{\it IBM T.J. Watson Research Center}\\
PO Box 218\\
Yorktown Heights, New York  10598
\end{center}

\vspace*{1.0in}

\begin{center}
{\bf Licensed Materials - Property of IBM\\}
{\bf \copyright Copyright IBM Corp. 2000  All Rights Reserved}
\end{center}


\newpage

\tableofcontents

\newpage

\listoffigures

\newpage

\section{Introduction}
\label{sect-intro}

{\bf DAISY} ({\bf D}ynamically {\bf A}rchitected {\bf I}nstruction
{\bf S}et from {\bf Y}orktown) translates a program at runtime from
one instruction set to another in a manner transparent to the user.
In other words, the user sees only the input instruction set.  This
implementation of {\bf DAISY} uses {\it PowerPC} as the input
instruction set and a new VLIW architecture as the target instruction
set.  This version also works on a page-by-page basis, translating all
code it can find on the first page of an application program, then
simulating this translated code until control passed to a new page,
then translating the second page, and so on.  Thus, if some code pages
are never executed, they are never translated.

\noindent Further (dated) information on {\bf DAISY} can also be found at the
DAISY website:

{\bf www.research.ibm.com/daisy}

\noindent and the IBM VLIW website:

{\bf www.research.ibm.com/vliw}

\section{Sample Session with DAISY}

We now describe {\bf DAISY} in more detail, beginning with a sample
session of a user running {\bf DAISY}.
Section~\ref{sect-internal-view} then gives a rough overview of what
{\bf DAISY} has done internally.

\subsection{User View}
\label{sect-user-view}

\begin{figure}
\begin{footnotesize}
\begin{verbatim}

Incorrect # of Arguments:
Correct usage is

	daisy [<flags>] <Executable> [<Executable Args>]

where <Executable> is a fully linked and executable XCOFF file, and
<Executable Args> are the normal arguments, if any, to <Executable>.
The <flags> are for "daisy".  Currently there are:

	-A:  Dump breakpoints for dbx in file "daisy.bk"

	-B<info_dir>:  Directory to put info files for "dvstats"
	               (Give full path if <Executable> changes curr directory)

	-C:  Run without DAISY translation.

	-E1: Trace all branches     in xlated   pgm.
	-E2: Trace branch-and-links in xlated   pgm.
	-E3: Trace cond branches    in xlated   pgm.
	-E4: Trace all branches     in xlated   pgm and libs.
	-E5: Trace branch-and-links in xlated   pgm and libs.
	-E6: Trace cond branches    in xlated   pgm and libs.

	-F:  Unroll loops as long as ILP improves.

	-G:  Perform cache simulation.
	-Z:  Save cache trace (as <Executable>.trace in <Executable> directory).

	-H<num1>: If DAISY compiler invoked <num1> times, clr xlation table.
	-J<num2>: If DAISY compiler invoked <num1> + <num2>
	          times, finish execution in native mode.

	-I:  Do NOT dump files needed by "dvstats"
	-N:  Dump VLIW code in "daisy.vliw"

	-K:  Override xlated program's segv handler

	-L<num>: Do -M action if hit <num> dyn crosspage entry pts
	-M<num>: If <num> = 1 stop, if 2 exec native code, else cont

	-P:  Support Power architecture instead of PowerPC

	-Q<num>:  If <num>==0, use real join pts, o.w. cnt only when pt starts
	          a continuation.  W/no -Q, cnt whenever instruc encountered.

	-R:  Use LOAD latency with LOAD-VERIFY, o.w. available immediately

	-S:  Do NOT split complex record ops: op. into op,cmpi
\end{verbatim}
\end{footnotesize}
\caption{(1) Usage info from invoking {\tt daisy} with no arguments.}
\label{fig-daisy-info-1}
\end{figure}

\begin{figure}
\begin{footnotesize}
\begin{verbatim}
Correct daisy usage flags (cont)

	-T:  Save xlation to disk (in ./save.daisy) for possible reuse)
	-U:  Retrieve prev xlation from disk (in ./save.daisy) for reuse)

	-V<num>: Running AIX version <num> (Default=414)

	-W:  Dump to stdout, a copy of stats appended to "daisy.stats"

	-X<num>:  Must have done "dvstats" on previous run
	  <num> = 0,1:  Use prof-dir-feedback w/cnts.
	  <num> = 2,3:  Use prof-dir-feedback w/prob.
	  <num> = 0,2:  Generate code for 0-count targets only if executed.
	  <num> = 1,3:  Always generate code for 0-count targets.

	-Y: STB r3,X ==> STB   r3,x          | STH r3,X ==> STH   r3,x
	    LBZ r4,X ==> RLINM r4,r3,0,24,31 | LHZ r4,X ==> RLINM r4,r3,0,16,31
	    No LD-VER needed, but copy-prop info for r3 not go to r4

Summary statistics are displayed to "stdout" and are
appended to the file "daisy.stats".  Cache simulation
results are placed in the file "cache.out", while the
cache configuration is read from the file "cache.config"
in the current directory or from the directory specified by
the CACHE_CONFIG environment variable.

Additional configuration parameters are read from the
file "daisy.config", if it exists.  For example "num_alus"
specifies the number of ALUS.  The format of "daisy.config"
is one quantity (e.g. "num_alus") per line followed by
whitespace and the numerical value, e.g. 16.  Blank lines
are ignored and # is a comment prefix for the line.  If
"daisy.config" exists in the current directory, it is used,
otherwise if "daisy.config" exists in the directory specified
by the environment variable DAISY_CONFIG, it is used.  Defaults
are used and a warning issued if neither is present.

The -L and -M can be very useful as debugging aids.
"daisy" can be run several times with the target program
using binary search to assign values to -L<num>.  For example use
-L1000 -M2 to see if the DAISY translated program crashes in the first
1000 page crossings.  If it does, then try -L500 -M2 in binary search
fashion.  Continue this until the precise page causing the crash is
located.  Assume the last value for which the program runs correctly
is -L319.  Then do an additional run with -L319 -M1 to determine the
PowerPC address of the page entry causing the problem.

NOTE:  The -H and -J flags used to be used for similar purposes, but
       use of -L and -M is much preferred.
\end{verbatim}
\end{footnotesize}
\caption{(2) Usage info from invoking {\tt daisy} with no arguments.}
\label{fig-daisy-info-2}
\end{figure}

The {\bf DAISY} executable is called {\tt daisy}.  A summary of {\tt
daisy} usage and its flags can be obtained by typing {\tt daisy} with
no arguments.  This summary is also reproduced in
Figures~\ref{fig-daisy-info-1} and~\ref{fig-daisy-info-2}.

In order to use {\tt daisy}, the {\tt datasize} limit must be at least
256 Mbytes.  Under {\it csh} or {\it tcsh}, this can be accomplished
via the command {\tt limit datasize 512000 kbytes}.  It is also useful
to have a large amount of paging space, at least 256 Mbytes, and
preferably 400 Mbytes.  The amount you currently have may be
determined by typing {\tt lsps -a}.  Contact your system administrator
if you need to increase your amount.

The most basic invocation of {\tt daisy} is to give it the name of the
executable which is to be translated and simulated.  For example, to
translate {\tt /bin/ls}, one types

\begin{verbatim}
    daisy /bin/ls
\end{verbatim}

The immediate result is the same as that from invoking {\tt /bin/ls}
directly, i.e. the contents of the current directory are displayed
on the screen, albeit more slowly.  In addition, {\tt daisy} appends
a variety of statistics about its translation of {\tt /bin/ls} to
a file called {\tt daisy.stats} in the current directory.  {\tt daisy}
also creates 3 additional files in the current directory:

\begin{verbatim}
   ls.vliw_perf_ins_cnts
   ls.vliw_perf_ins_info
   ls.vliw_spec_ins_info
\end{verbatim}

These binary files are used to create a myriad of statistics using a
post-pass tool called {\tt dvstats}.  {\tt dvstats} takes as 
arguments the base name of the translated program, as well as the
directory in which the 3 files above may be found.  For example,

\begin{verbatim}
   dvstats ls .
\end{verbatim}

\noindent The final ``.'' specifies that dvstats should look in the
current directory for the 3 files listed above.  Running {\tt dvstats}
produces several more files, all with the prefix {\tt ls.}.  Probably
the most interesting is {\tt ls.histo} which has the infinite cache
ILP attained by the translated code.  Infinite cache ILP is measured
simply as the ratio of {\it PowerPC} instructions used to execute the
original program ({\tt /bin/ls}) to the number of VLIW instructions
used to execute the original program.  The {\tt ls.histo} file also
reports how many and what type of operations were packed into each
VLIW instruction.  Other files report opcode frequencies and critical
paths through the code, along with the ILP attained on each of those
paths.  These matters are discussed in more detail in
Section~\ref{sect-post-proc}.

One other file is of significant interest here.  If the {\tt -N} flag
is given with {\tt daisy}, a dump of the VLIW code is produced in a
file called {\tt daisy.vliw}.  This file can be more than a megabyte in
size, even for a program like {\tt /bin/ls} because shared library
code (e.g. {\bf printf} and {\bf malloc}) is translated too, and
because there is a significant explosion in code size -- typically a
factor of 4, but varying widely.  Note too that any previous {\tt
daisy.vliw} file in the current directory is destroyed when the {\tt
-N} flag is specified.  Section~\ref{sect-vliw-dump} gives a
description of the syntax of the VLIW instructions in {\tt daisy.vliw}

\subsection{Internal View}
\label{sect-internal-view}

\begin{figure}
\begin{center}
\ \psfig{figure=fig/daisy_cg1.eps}
\end{center}
\caption{Initialization, configuration, and start of translation under
{\tt daisy}.}
\label{fig-daisy-init-cg}
\end{figure}

\begin{figure}
\begin{center}
\ \psfig{figure=fig/daisy_cg2.eps}
\end{center}
\caption{Steady state of {\tt daisy} mixing execution and translation.}
\label{fig-daisy-steady-cg}
\end{figure}

Figures~\ref{fig-daisy-init-cg} and~\ref{fig-daisy-steady-cg} depict
high level call-graphs of what {\tt daisy} does internally as it
translates {\tt /bin/ls}.  Figure~\ref{fig-daisy-init-cg} depicts how
{\tt daisy} initializes and configures itself, then begins
translating.  Figure~\ref{fig-daisy-steady-cg} depicts {\tt daisy} in
the steady-state when it is mixing translation and execution of {\tt
/bin/ls}.  Note that the shaded portion of
Figures~\ref{fig-daisy-init-cg} and~\ref{fig-daisy-steady-cg} is
common and corresponds to the actual work of translation.  Below is a
fuller description of the internal processing depicted in
Figures~\ref{fig-daisy-init-cg} and~\ref{fig-daisy-steady-cg}.

\begin{itemize}

  \item The {\bf main()} function is in {\tt main.c} (Actually {\bf
	main2()} is in {\tt main.c}.  {\bf main()} is in {\tt
	r13\_base.s} so as to properly setup the stack for the
	translated program, e.g. {\tt /bin/ls}.  The value of the
	stack to use for the translated program is the {\tt sp\_val}
	parameter passed to {\bf main2()} by {\bf main()}.)

  \item {\bf main2()}, first checks for any flags such as {\tt -N} via
	a call to {\bf handle\_flags()} in {\tt flags.c}.

  \item {\bf main2()} then opens the file to be translated (e.g {\tt
	/bin/ls}) and reads the header in order to find the number
	of instructions in the text section.  (This is mainly
	used for informational purposes to know what fraction of
	the execution time was spent in shared libraries.)

  \item {\bf main2()} then loads the target program into memory via a
	call to {\bf load\_prog\_to\_exec()}, which in turn invokes the
	AIX function {\bf load()}, which reads the target XCOFF file and
	performs all necessary relocation.

  \item {\bf main2()} then does a variety of initializations via the
	call to {\tt init\_everything()}.  One of these
	initializations is to call {\tt vliw\_init()} in {\tt vliw.c}.
	{\tt vliw\_init()} in turn reads additional configuration
	information such as the number of registers and function units
	via a call to {\bf read\_config()} in {\tt rd\_config.c},
	which in turn reads this information from the file {\tt
	daisy.config}.  If the environment variable, {\it
	DAISY\_CONFIG} is set, {\bf read\_config()} assumes it
	specifies the directory in which to look for {\tt
	daisy.config}.  If {\it DAISY\_CONFIG} is not set, then {\bf
	read\_config()} looks for {\tt daisy.config} in the current
	directory.  If {\tt daisy.config} does not exist, {\bf
	read\_config()} provides a default configuration.

  \item Note that {\bf DAISY} has its own version of {\bf malloc()}.
	If {\bf DAISY} invokes {\bf malloc()}, it perturbs how the
	translated program runs.

  \item {\bf main2()} then does various initializations dealing with
	profile-directed feedback and tracing the branches in the
	original {\it PowerPC} program (e.g. {\tt /bin/ls}).

  \item Next {\bf main2()} makes sure that the VLIW registers
	match the {\it PowerPC} registers at the start of the program,
	e.g.  VLIW {\tt r3} and {\it PowerPC} {\tt r3} both contain
	{\bf argc}.  The call to {\bf setup\_regs()} which does this,
	also sets up several other values for use during simulation.
	Section~\ref{sect-vliw-dump} discusses this in more detail.

  \item In order to run correctly, the translator and the
	translated/simulated program must use different stack
	pointers.  Thus {\bf main2()} sets up the simulator
	(and hence the translated program (e.g {\tt /bin/ls})) to use
	the translator's stack pointer ({\tt sp\_val}).  When the
	translator is re-invoked to translate subsequent pages it will
	use its own stack pointer pointing to a statically allocated
	stack.  This is done because we know that the translator can
	make do with a small fixed stack size, but we do not know if
	this is true of the translated program.

  \item {\bf main2()} then sets up a couple of additional values
	for use by the simulator in {\bf r13\_area} and invokes {\bf
	xlate\_entry()}.

  \item As depicted in Figure~\ref{fig-daisy-steady-cg}, {\bf
	xlate\_entry()} is the main entry point of the translator, and
	takes one argument, the address from which it is to translate.
	Subsequently when a page needs translation, the simulator
	branches to {\bf xlate\_entry\_raw()}, in {\tt r13\_base.s}.
	{\bf xlate\_entry\_raw()} (1) saves registers live during the
	execution of simulated VLIW code, (2) sets up registers and
	the stack frame for use by normal C code such as the {\bf
	xlate\_entry()} function, and (3) calls {\bf xlate\_entry()}.
	When {\bf xlate\_entry()} returns, it passes in register {\tt
	r3} the address of the simulation code for the VLIW code just
	generated.  {\bf xlate\_entry\_raw()} then restore registers
	used during exeuction of simulated VLIW code and then branches
	to the address passed in register {\tt r3}.

  \item {\bf xlate\_entry()} does some initialization of its own, then
	enters a doubly nested {\tt while} loop.  The outer {\tt
	while} loop checks via a call to {\bf
	get\_highest\_prio\_cont()} in {\tt contin.c}, if there are any
	open paths on the page that remain to be translated.
	Initially there is one such open path --- the entry point to
	the page.  Translation on a path stops whenever a branch is
	encountered.  If the branch is conditional, two new path
	continuations --- the target and the fall-through --- are
	added to the open path list (assuming that both are on the
	current page).  Additions to the open path list are made via
	calls to {\bf add\_to\_xlate\_list()}.  Unconditional branches
	of course place only their target in the path list~\footnote{
	{\tt Branch-and-link} instructions can optionally place the
	fall-through path in the list as well.}.

  \item Once a path continuation is found, several checks are made as
	to whether {\it PowerPC} operations from this path should be
	scheduled into earlier VLIW instructions --- thus increasing
	parallelism, or if a new scheduling region should begin at the
	current operation --- thus decreasing code explosion.

  \item In either case, scheduling of {\it PowerPC} operations from the
	current path is done in the inner {\bf while (TRUE)} loop.

  \item A call to {\bf get\_opcode()} partially parses the {\it PowerPC}
	operation.

  \item Then several special types of {\it PowerPC} operation are
	checked for --- {\it illegal ops, load/store multiple ops,}
	and {\it load/store update ops.}

  \item Scheduling on this path stops if there is an {\it illegal op.}
	If execution comes here, something is probably wrong with the
	original program.  If this path is actually executed, the
	simulator will print a message that an {\it illegal op} was
	encountered and terminate the run.

  \item {\it Load/store multiple} and {\it load/store update}
	operations are split into simpler primitives and then
	scheduled like any other operation.  There is one caveat to
	this.  {\tt lwsx}, {\tt stwsx}, (and {\tt lscbx}) are not
	split.  Instead {\bf DAISY} acts as if these were VLIW
	primitive operations.  {\it This should change at some point!}

  \item For other {\it PowerPC} operations, parsing is completed by a call
	to {\bf set\_operands()}.  Another illegal check must be
	made here because some operands of {\tt mtspr} and {\tt mfspr}
	are illegal, while others are not.

  \item Legal opcodes are sent to the routine {\bf
	xlate\_opcode\_p2v()}.

  \item {\bf xlate\_opcode\_p2v()} divides instructions into {\it
	branches} and {\it ALU ops}.

  \item A variety of optimizations can be done with ALU and memory
	operations.  These include {\it copy propagation, combining,
	load-motion above stores, load-store alias analysis,} and {\it
	dead-code elimination.}  All of these are checked by {\bf
	xlate\_opcode\_p2v()}.

  \item Additional splitting of complex ALU ops into simpler ones is
	done in {\bf xlate\_opcode\_p2v()}.  In particular {\it
	record} form ops (e.g. {\tt and.}) that set {\bf CR0} are
	split (e.g. to {\tt and} followed by {\tt cmpli}).  Likewise
	{\tt rlwimi} is split and the individual pieces scheduled.

  \item If none of the optimizations or splittings apply --- the most
	common case --- {\bf xlate\_opcode\_p2v()} calculates the
	earliest time at which the op may be scheduled due to data
	dependences.  This is done via a call to {\bf
	get\_earliest\_time()}, which is in {\tt vliw.c}.  {\bf
	xlate\_opcode\_p2v()} then calls {\bf insert\_op()} to
	schedule the operation at or after this earliest time, as
	function unit and register availability allow.  Exactly how
	{\tt insert\_op} does scheduling is discussed in
	Section~\ref{sect-sched-alg}.

  \item Branch operations are split into several different types, the
	main being {\it direct onpage branches}, {\it direct offpage
	branches}, and {\it indirect branches} through the {\tt
	Counter} or {\tt LinkReg}.  {\tt RFI} -- return from
	interrupt, {\tt SC} (or {\tt SVC}), and {\tt traps} are also
	handled, although {\tt SC} and {\tt traps} are infrequent in
	user code and {\tt RFI} is non-existent.

  \item Returning up the call chain to {\bf xlate\_entry()}, once
	all operations on all open paths are scheduled, the outer
	{\tt while} loop exits.

  \item {\bf xlate\_entry()} then checks if the user wants a dump of the
	VLIW code.  The {\tt -N} flag discussed above in
	Section~\ref{sect-user-view} corresponds to the {\bf
	do\_vliw\_dump} variable.

  \item In order to generate statistics at the end of a program run,
	{\bf DAISY} must know when a program ends.  Programs usually
	end via a kernel service call to location {\bf 0x3600} (or
	{\bf 0x37D0} in AIX 4.3).  The value in register {\tt r2}
	determines the service requested.  Thus when the {\bf DAISY}
	simulator sees that it is executing the translation of
	location 0x3600, it checks if {\tt r2} contains the {\bf
	\_exit()} service value.  If so, it writes its statistics to
	disk, as described above in Section~\ref{sect-user-view}, and
	then terminates.  The call to {\bf dump\_exit\_chk()} sets up
	the simulator to perform this check.

  \item The {\bf DAISY} simulator is actually a translation of the
	VLIW code back to {\it PowerPC} operations which then run
	natively on the {\it PowerPC} machine.  This is detailed in
	Section~\ref{sect-simul-code}. The call to {\bf
	vliw\_to\_ppc\_simcode()} performs this translation.  The
	simulation code not only models the execution of VLIW
	instructions, but also keeps count of how many instructions
	execute, and can also call a cache simulator.

  \item We want subsequent branches to the entry point of this page to
	go to the translation we just created, not to the translator.
	In order to do this {\bf DAISY} maintains a hash table mapping
	{\it PowerPC} instruction address to their corresponding VLIW
	translation.  The call to {\bf add\_page\_hash\_entries()}
	creates this mapping.  Note that not only the entry point
	of the page goes into this hash table, but any other points
	at which a new scheduling region is begun, as was described
	above.

  \item After this, {\bf xlate\_entry()} returns to the {\bf main2()}
	function.  {\bf main2()} knows that simulated code for the
	first group -- the entry point of a program is always
	at the start of {\bf xlate\_mem} -- the start of translation
	memory, and sets up an AIX-style function pointer so that
	an indirect branch can made to this first translation.

  \item {\bf main2()} then branches to this address to begin
	executing the translated program (e.g {\tt /bin/ls}).

  \item When the translated program comes to a new page or performs
	an indirect jump, it will re-invoke {\bf xlate\_entry()} to
	do the same tasks we just outlined for the first page.

  \item Subsequent invocations of {\bf xlate\_entry()} actually occur
	indirectly, as can be seen in
	Figure~\ref{fig-daisy-steady-cg}.  The simulation code
	branches to {\bf xlate\_entry\_raw()} which is in {\tt
	r13\_base.s}.

  \item {\bf xlate\_entry\_raw()} saves {\it PowerPC} registers that
	may be live (e.g. {\tt r1-r12}, {\tt fp0-fp3}, and the {\tt
	XER}).  It then sets up the TOC (register {\tt r2}, which is
	used as a {\it Table Of Contents} in AIX to access global
	variables) and stack pointer ({\tt r1}) for the translator.
	It also copies parameters passed by the simulator to standard
	AIX parameter registers.  For example the address to
	translate is copied from {\tt r27} to {\tt r3}.  Upon return
	to {\bf xlate\_entry\_raw()}, control is returned to the
	simulated VLIW code, as described above.

  \item This process continues until the simulation code detects the
	translated program is exiting, as described above.

\end{itemize}

\section{VLIW Tree Instructions and Generic Machine Model}
\label{sect-tree-ins-mach-model}

\begin{figure}
\begin{center}
\ \psfig{figure=fig/tree_ins.eps}
\end{center}
\caption{Sample VLIW Tree Instruction.}
\label{fig-vliw-tree}
\end{figure}

{\bf DAISY} currently targets a VLIW machine whose instructions are
trees, a format developed by Ebcioglu~\cite{Ebcioglu88} and his
collaborators over the years.  A sample tree is shown in
Figure~\ref{fig-vliw-tree}.  There are several points to note:

\begin{itemize}

  \item The tree instruction splits at conditional branch
	instructions.  The left path corresponds to the branch
	condition being false, while the right path corresponds
	to the branch condition being true.  In
	Figure~\ref{fig-vliw-tree} there are two conditional
	branches.  The edge terminating at {\tt b V2}
	corresponds to the false path of {\tt bc CR0.eq}.  The
	true path of {\tt bc CR0.eq} splits at the branch {\tt
	bc CR1.lt} with the path ending at {\tt b V7}
	corresponding to the false path of {\tt bc CR1.lt} and
	the path ending at {\tt b V9} corresponding to the true
	path.

  \item The branch conditions are evaluated prior to executing
	the tree.  For example, the {\tt bc CR0.eq} uses the
	value of {\tt CR0.eq} that existed upon entry at the
	label {\tt V1}, not the value computed by {\tt cmp
	CR0,r4,r5}.

  \item With one exception, all input values (like {\tt CR0.eq}
	in {\tt bc CR0.eq} or {\tt r4} and {\tt r5} in {\tt add
	r3,r4,r5}) are evaluated prior to the execution of the
	VLIW instruction.  This is termed {\it parallel
	semantics}, i.e. all the inputs are read in parallel at
	the start of the VLIW instruction, and all the outputs
	are written in parallel at the end of the VLIW
	instruction.

  \item {\it Aside:} Parallel semantics are not essential to
	VLIW tree instructions.  Sequential semantics in which
	each operation executes and writes its results before
	the next executes are also possible.  However sequential
	semantics restrict how VLIW instructions may be formed.
	For example, it would not be legal to write {\tt CR0}
	via the {\tt cmp} and read it via {\tt bc CR0.eq} in the
	same VLIW, as occurs in Figure~\ref{fig-vliw-tree}.
	Sequential semantics do have an advantage.  They allow
	large VLIW tree instructions to be easily split into
	smaller trees in hardware if the machine on which they
	are running does not have sufficient function units.

  \item The one exception to parallel semantics in {\bf DAISY} is
	memory references.  Since the compiler cannot always
	know if a store and load are independent, it must have
	some way of guaranteeing that the load receives the
	correct value, and that memory is updated correctly and
	in the proper order.  One way to do this would be (1) to
	allow at most one store per VLIW instruction and (2) to
	allow no loads ``after'' any store in a VLIW
	instruction.  Unfortunately, these restrictions
	significantly reduce the amount of instruction level
	parallelism achievable.  Thus {\bf DAISY} leaves it to the
	hardware to insure that stores are properly sequenced to
	memory and that loads receive the proper value, even if
	it is from a store in the same VLIW instruction.  The
	hardware may introduce stalls if there is too much such
	aliasing, so it is still useful for the compiler to do
	as much disambiguation as possible.  (Other
	optimizations are also possible from such disambiguation
	work.)

  \item Since the tree in Figure~\ref{fig-vliw-tree} has two
	conditional branches, there are three exits, {\tt b V2},
	{\tt b V7}, and {\tt b V9}.  In general, there is one
	more exit than there are conditional branches.

  \item ALU and memory operations are associated with edges of
	the tree.  In Figure~\ref{fig-vliw-tree}, there are 5
	edges.  Whether ALU/memory operations on each edge
	actually execute depends on the evaluation of the
	conditional branches.  If an operation is on an edge
	that matches conditional branch conditions, it is
	executed, otherwise not.  For example, the {\tt and
	r6,r5,r4} on the edge from {\tt bc CR0.eq} to {\tt b V2}
	is executed only if {\tt CR0.eq} is not set.

  \item In this way, the conditional branches in a VLIW
	instruction act not so much as branches, but as masks
	indicating which set of instructions in the VLIW tree
	are to execute.  Since the branch conditions are
	evaluated at the start of the VLIW instruction, the set
	of ALU and memory operations to be executed can be
	determined at the start of the instruction as well.

  \item As a practical matter, each VLIW tree instruction executes in
	one cycle, which does not leave sufficient time --- at least
	at the frequencies we wish to achieve --- to evaluate which
	operations should execute, and then execute them.  Thus all
	operations in any likely implementation are executed and the
	mask determined by the conditional branches is used to enable
	write-back of those results which actually should have
	executed.

  \item As can be seen in Figure~\ref{fig-vliw-tree}, each VLIW
	instruction explicitly identifies its successor
	instructions, i.e. VLIW {\tt V2}, {\tt V7}, or {\tt V9}.
	Thus VLIW instructions branch on every cycle.  There
	are two problems with such branching:

  \begin{itemize}
    \item Encoding 3 (or 4) completely independent targets in a 
	  VLIW instruction is wasteful of encoding space.

    \item Fetching new instructions from a new and arbitrary
	  location every cycle is difficult, all the more so if
	  the conditional branch conditions must be computed to
	  determine the next VLIW instruction to fetch.
  \end{itemize}

	To overcome these difficulties, the eventual {\bf DAISY}
	architecture will almost certainly place restrictions,
	on the set of successor VLIW instructions to the current
	VLIW instruction.  One likely restriction is that all
	successor instructions lie in the same Icache line.  If
	this is the case, the fetch unit can begin fetching this
	cache line immediately without waiting to see how the
	condition codes evaluate.  Then at the end of the cycle,
	when the condition codes have been evaluated, it can
	pick the proper successor instruction out of the cache
	line.  {\it The {\bf DAISY} translation software currently
	does not take into account such restrictions.} This
	shortcoming needs to be fixed, possibly through a
	``final-pass'' assembler which organizes the VLIW
	instructions --- duplicating them in some cases --- in
	order to meet the cache line restriction.

  \item Note that register {\tt r3} is written twice if the path
	from {\tt V1} to {\tt b V7} is followed.  In such cases,
	the operation nearer the VLIW exit takes precedence.
	Thus, if the {\tt V1} to {\tt b V7} path is followed,
	the {\tt nor r3,r8,r7} result takes precedence over the
	{\tt add r3,r4,r5} result.
	
\end{itemize}

\begin{figure}
\begin{verbatim}
               add   r3,r5,r5,<TRUE>
               cmp   CR0,r4,r5,<TRUE>
               and   r6,r5,r4,<CR0.eq FALSE>
               or    r6,r5,r4,<CR0.eq TRUE>
               nor   r3,r8,r7,<CR0.eq TRUE, CR1.lt FALSE>
               xor   r5,r6,r9,<CR0.eq TRUE, CR1.lt TRUE>
               bc    <CR0.eq FALSE>,V2
               bc    <CR0.eq TRUE, CR1.lt FALSE>,V7
               bc    <CR0.eq TRUE, CR1.lt TRUE>,V9
\end{verbatim}
\caption{Sample Predicated Code Corresponding to VLIW Tree Instruction.}
\label{fig-vliw-pred}
\end{figure}

Note that VLIW tree instructions are quite similar to predicated
execution, with the conditional branches in the VLIW tree
instruction acting as predicates.  For example,
Figure~\ref{fig-vliw-pred} shows equivalent predicated code for
the VLIW tree instruction in Figure~\ref{fig-vliw-tree}.  Two
points are of particular note:

\begin{itemize}

  \item Operations may have multiple predicates. For example the
	{\tt xor} operation from the VLIW tree instruction in
	Figure~\ref{fig-vliw-tree} has two predicates, {\tt
	<CR0.eq TRUE, CR1.lt TRUE>}, in
	Figure~\ref{fig-vliw-pred}.

  \item Branches to the next VLIW tree instruction become
	conditional branches, again possibly with multiple
	conditions, for example, {\tt bc <CR0.eq TRUE, CR1.lt
	TRUE>,V9} in Figure~\ref{fig-vliw-pred}.

\end{itemize}

\noindent {\it Another possible task in the {\bf DAISY} compiler} is
to make it support an architecture with explicitly predicated
operations, as in Figure~\ref{fig-vliw-pred}.

There is of course more to the {\bf DAISY} machine model than simply
the instruction model.  Many of these other parameters, however
are more easily parameterized:

\begin{itemize}

  \item The {\bf DAISY} compiler currently supports machines having
	from 64 to 256 general purpose registers, 64 to 256 floating
	point registers, and 64 to 256 condition register bits (or
	equivalently for {\it PowerPC}, 16 to 64 {\it 4-bit} condition
	register fields).  The maximum number is specified in the file
	{\tt max\_resrcs.h} and the number used during translation can
	be set in the {\tt daisy.config} file as described in
	Section~\ref{sect-internal-view}.  ({\bf Caution:} {\it There
	are currently bugs if more than 64 of any one kind of
	registers are specified.})

  \item The {\bf DAISY} compiler currently supports machines having up
	to 16 ALU and/or memory operations plus 8 conditional
	branches per VLIW tree instruction.  (Larger numbers
	should be possible, as these values are generally
	parameterized in the code, however there seems to be
	some bug $\ldots$.)  The exact numbers for a particular
	run are also set in {\tt daisy.config}.

  \item Function units can be clustered, so that results
	computed in a cluster are available as inputs in the
	next cycle within that cluster, but only after a 1 cycle
	delay in other clusters.  The eventual {\bf DAISY}
	architecture will likely be clustered, and the software
	may need to be modified to support the precise model
	adopted.  Much of the code for clustering is in {\tt
	cluster.c}.  The clustering parameters are set in {\tt
	daisy.config} and may be changed for each translation.

  \item This version of {\bf DAISY}  does not distinguish between ALU types,
	and in particular does not distinguish between integer
	and floating point operations.

  \item Instruction latencies are configured at compile time,
	not execution time.  (That is they are configured when
	{\bf DAISY} is compiled, not when {\bf DAISY} is run.)  Several sets
	of latencies have been used, and these are controlled by
	{\tt \#ifdef} statements in the file {\tt latency.h}.

  \item {\bf DAISY} can model up to 3-level cache and TLB hierarchies.
	Specifying the {\tt -G} flag with {\tt daisy} enables cache
	simulation.  When cache simulation is enabled, {\bf DAISY}
	reads the cache configuration from the file {\tt cache.config}
	and the TLB configuration from the file {\tt tlb.config}.
	Both {\tt cache.config} and {\tt tlb.config} are picked up
	from the directory specified by {\tt DAISY\_CONFIG} or from the
	current directory, just as is the case with {\tt daisy.config}.
	The cache and TLB simulation functions are the same and are
	mostly in the file {\tt cache\_simul.c}.  The TLB's just look
	like an independent cache hierarchy with 4K lines.  Cache
	statistics are written to the file {\tt cache.out}, while TLB
	statistics are written to the file {\tt tlb.out}.
	

\end{itemize}

\section{{\bf DAISY} Scheduling Algorithm}
\label{sect-sched-alg}

\begin{figure}

\begin{minipage}[t]{2.9in}
\begin{center}
\ \psfig{figure=fig/p2v-1.eps,width=2.7in,height=3.3in}
\end{center}
\end{minipage}
%%
\hspace*{0.05in}
%%
\begin{minipage}[t]{2.9in}
\begin{center}
\ \psfig{figure=fig/p2v-2.eps,width=2.7in,height=3.3in}
\end{center}
\end{minipage}
\caption{Example of conversion from {\it
PowerPC} code to VLIW tree instructions.}
\label{fig-ppc-to-vliw}
\end{figure}

\begin{figure}[htb]
\begin{center}
\ \psfig{figure=texfig/p2v-txt.eps,height=8.00in}
\end{center}
\caption{Description of conversion from {\it PowerPC} to VLIW.}
\label{fig-ppc-to-VLIW-expl}
\end{figure}

The scheduling algorithm is at the heart of {\bf DAISY} translation
from {\it PowerPC} code to VLIW code, as it is the basic mechanism for
extracting instruction level parallelism from {\it PowerPC} code.  The
algorithm is illustrated in Figure~\ref{fig-ppc-to-vliw}.  A detailed
description of why each scheduling step is performed is contained in
Figure~\ref{fig-ppc-to-VLIW-expl}.  However, there are three major
points:
%%
  \begin{itemize}

  \item Operations 1--11 of the original {\it PowerPC} code are
	scheduled in sequence into VLIW's.  It turns out that two
	VLIW's suffice for these 11 instructions.

  \item Operations are always added to the end of the last VLIW
	on the current path.  If input data for an operation are
	available prior to the end of the last VLIW, then the
	operation is performed as early as possible with the
	result placed in a renamed register (that is not
	architected in the {\it PowerPC}).  The renamed register is
	then copied to the original {\it PowerPC} register at the end
	of the last VLIW.  This is illustrated by the {\tt xor}
	instruction in step 4, whose result is renamed to {\tt
	r63} in VLIW1, then copied to the original destination
	{\tt r4} in VLIW2.  By having the result available early
	in {\tt r63}, later instructions can be moved up.  For
	example, the {\tt cntlz} in step 11 can use the result
	in {\tt r63} before it has been copied to {\tt r4}.
	(Note that the scheduler assumes the parallel semantics
	described in Section~\ref{sect-tree-ins-mach-model}, in
	which all operations in a VLIW instruction read their
	inputs before any outputs from the current VLIW
	instruction are written.)

  \item The renaming scheme just described places results in
	architected {\it PowerPC} registers in original program order.
	Stores and other operations with non-renameable
	destinations are placed at the end of the last VLIW on
	the current path.  In this way, precise exceptions can
	be maintained.

  \end{itemize}
%%

The {\bf DAISY} code for performing this scheduling is contained
largely in the file {\tt vliw.c}, and in particular, the
function {\bf insert\_op()}.  {\bf insert\_op()} is passed a
{\it PowerPC} operation to schedule and an earliest time at which that
operation may be scheduled.  To get this earliest time, the
caller of {\bf insert\_op()} typically calls the function {\bf
get\_earliest\_time()}, which is also in {\tt vliw.c}.  {\bf
get\_earliest\_time()} returns the earliest time at which the
current {\it PowerPC} operation may be scheduled based on data
dependences with already scheduled operations.

The first VLIW in a group has time 0, the subsequent VLIW(s),
time 1, etc.  There may be multiple immediate successors if the
fist VLIW instruction contains a conditional branch.  {\bf DAISY} does
not assume that the hardware has interlocks and if it encounters
a long latency operation such as {\tt divide} and no useful work
to be done between the {\tt divide} and its use, it pads the gap
with empty VLIW instructions.

{\bf insert\_op()} first tries to place the {\it PowerPC} operation at
the end of the VLIW instruction at the {\it earliest time}.  To
do this, {\bf DAISY} requires (1) that a function unit be free on
which to compute the result, and (2) that a non-{\it PowerPC}
architected register be available to place the result if the
{\it earliest time} is earlier than the current VLIW
instruction.  If these function unit or register constraints do
not permit placing the operation at the {\it earliest time}, the
next time is tried, until eventually a VLIW instruction is found
in which the {\it PowerPC} operation may be scheduled.  This is
guaranteed to happen at some point, because if all of the
existing VLIW instructions on the path are full, {\bf
insert\_op()} creates a new empty VLIW instruction and appends
it to the current path.  Note that this scheduling algorithm
never needs to spill registers, since having a register for the
result is a requirement of scheduling an operation.  Of course,
if the {\it PowerPC} operation is scheduled at the end of the current
VLIW instruction or later, it immediately places result in the
{\it PowerPC} destination register.

Although it may not be clear from this example, the scheduling
algorithm produces groups of VLIW tree instructions which are
themselves trees.  That is the groups are trees of trees.  The
reason for this is in the second bullet above: namely {\it PowerPC}
operations or their {\it commit} from a rename register are
always scheduled in the last VLIW instruction on the current
path.  Even if the {\it PowerPC} operation has been seen before on
some other path, it is replicated again at the end of the
current path.  Thus VLIW groups have no {\it join} points.

Sometimes, of course, it is essential to have a join point.
Loops for example need a join at their entry, or the scheduling
algorithm would unroll them forever.  This is accomplished
by terminating the current path and starting a new VLIW group
after a particular {\it PowerPC} operation has been seen sufficiently
many times.

The data structures used by the scheduling algorithm, in
particular the representation of VLIW tree instructions is
described in Section~\ref{sect-data-struc}.

\section{{\bf DAISY} Optimizations}

\subsection{Combining}

The combining optimization essentially does symbolic
arithmetic, and is very useful for noting when address
expressions are the same, as well as for generating induction
variable values in {\tt for} loops, without serialization.
For example, a loop body may have a statement like {\tt addi
r3,r3,4}.  Thus if {\tt r3} has the value {\tt r3}$_0$ in
iteration 0, then in iteration 3, {\tt r3} will have the value
{\tt r3}$_0+12$.  Combining determines this and eliminates
serialization on the {\tt addi r3,r3,4} instruction.  This
then allows iteration 3 operations, which have no other
data dependences to be scheduled immediately.

Address expressions, particularly those involving the stack
also benefit from combining.  The stack is often bumped at the
start of a procedure, e.g. {\tt addi r1,r1,-96}.  This may be
followed by a load operation, e.g. {\tt lwz r4,8(r1)}.
Combining notes that this load operation can be scheduled
immediately (before the {\tt addi}) as {\tt lwz r4,-88(r1)}.

There are some subtle caveats on the use of combining.  For
example, there are some cases when wrap-around from 0xFFFFFFFF
to 0x00000000 makes combining impermissible.  One such case
involves {\tt addic}.  For example, assume that {\tt r4} is
initially 0xFFFFFFFF, and a loop has the instruction {\tt
addic r4,r4,1}.  Then the first time that this instruction is
executed, the {\tt CA} bit of the XER is set since the new
value wraps around to 0x00000000.  However, the second
execution of this instruction clears the {\tt CA} bit since
there is no carry going from 0 to 1.  However, combining might
rewrite this second iteration as {\tt addic r63,r4,2}, which
would incorrectly set the {\tt CA} bit.

DAISY currently implements combining by maintaining a linked
list of combined values for each GPR.  Each entry in this list
contains

\begin{itemize}

  \item A GPR number,
  \item A time,
  \item A displacement,
  \item A pointer to the next list element.

\end{itemize}

This list is kept in reverse chronological order, and thus to
find the proper base register and displacement at a particular
time, the list is searched until the {\it time} is less than
or equal to the time of interest.  The desired value of the
current GPR is then obtained from the sum of the {\it GPR
number} and {\it displacement} in the list.  This linked list
is added to whenever a combinable operation is encountered.
For example, {\tt addi} and {\tt oril rx,rx,0} (or any {\tt
copy}) are combinable operations.  This list is associated with
the {\bf TIP} structure and is called {\tt gpr\_val}.

\subsection{Copy Propagation}

Copy propagation is essentially a special case of combining
where the displacement must always be 0.  However, it is useful
to separate the two, because not all operations have a
displacement field.  For example, the operation {\tt xor
r3,r4,r5} is not helped if we know that at a particular time
{\tt r4} is equivalent to {\tt r63+8}.  However, if we know
that {\tt r4} is equivalent to {\tt r61}, then we can schedule
the {\tt xor} early as {\tt xor r60,r61,r5} (assuming that
{\tt r60} is an available destination register and that {\tt
r5} maps to itself at the time the {\tt xor} is scheduled.
Copy propagation information is kept for each GPR, FPR, and
CCR in a linked list akin to that used for combining.  The
only difference is that there is no displacement field.  These
lists are part of the {\bf TIP} structure and are called {\tt
gpr\_rename}, {\tt fpr\_rename}, and {\tt ccr\_rename}.

\subsection{Load-Store Must Aliases}
\label{sect-must-alias}

There are many cases, particularly in unoptimized code where
it can be easily proven that a particular load refers to the
same location as a previous store encountered on this path.
For example:

\begin{verbatim}
   xor    r3,r9,r10
   stw    r3,8(r1)
   ...
   lwz    r5,8(r1)
   addi   r6,r5,1
\end{verbatim}

In such cases, operations depending on the result of the {\tt
load} can be aggressively scheduled using the source of the
{\tt store} operation.  In the example above, the {\tt addi}
instruction could be scheduled immediately after the {\tt xor}
as {\tt addi r63,r3,1}, i.e. the {\tt addi} need not serialize
on the memory accesses.  This is true even if there are other
stores between the {\it must-alias} load and store.  If some
such store writes to the same memory location, then whatever
method is used to ensure memory consistency (e.g. load-verify)
will raise an exception, and the {\tt addi} result in {\tt
r63} will be discarded.  However, in the normal case, where
there is no such problem store, the {\tt addi} is computed far
earlier than it was in the original code.  We sometimes refer
to this optimization as {\it load-store telescoping}.

These load-store {\it must-aliases} benefit significantly from
combining and copy-propagation.  Combining and copy-propagation
make it easy to determine if address expressions are equivalent
even if they do not refer to the same register.  Furthermore,
{\it load-store telescoping} can be done through multiple sets
of stores and loads.

{\bf DAISY} implements its {\it must-alias} analysis via the
{\tt stores} field of the {\bf TIP} structure.  The {\tt stores}
field is a list of all the {\tt stores} which have occurred on
this path through the group.  Whenever a load is encountered,
its address is compared to those in {\tt stores} for a match.
In the case of multiple matches, the latest match, of course
wins.

\subsection{Load-Verify}
\label{sect-load-verif}

In order to get reasonable ILP, it is essential to execute load
instructions as early as possible, and not just in those cases
where {\it must-alias} can be proven.  Since the translator
cannot be certain when the memory location is ready, {\it as
early as possible} in this context means when all of the
register inputs to the load are ready.  (There are exceptions
such as {\tt sync} operations which order memory accesses, but
we will ignore those here.)

Such speculative scheduling of loads requires some means to
check whether the loaded value was correct.  The {\tt load-verify}
instruction is one way of performing this check.  Whenever a load
operation is scheduled speculatively (and moves past at least one
store), a {\tt load-verify} instruction is inserted in the original
order.  This {\tt load-verify} instruction reloads the value
and checks whether it matches the original value.  If so, execution
continues normally.  If not, a {\tt load-verify exception trap} 
occurs, and the {\bf DAISY} VMM retranslates starting at the problem
load.  When this new translation is executed, the load executes
first and gets the correct value.

If a particular {\tt load} repeatedly incurs such traps, {\bf
DAISY} retranslates the original group without speculating the
load instruction.  Currently {\bf DAISY} does this if a
particular load traps 10 times.  Such a retranslate policy
drops the number of {\tt load-verify traps} from hundreds of
millions to hundreds in the SPEC95 benchmarks.

Memory mapped I/O locations present a particular problem for
speculative loads.  Such locations normally cannot be read or
written speculatively because they can have real physical side
effects such as displaying a character on the screen or
launching a missile.  (Even loads can have such side effects.
For example some I/O cards have only 1 address, and in order
to read multiple values from that address, sequence through
the set of values read from that address.)  Thus, any hardware
on which speculative loads are supported must be able to
recognize such I/O accesses and quash them.  On {\it PowerPC},
this can be done via the WIMG bits, for example.

\subsection{Dead Code Elimination}

{\bf DAISY} employs a very simple and limited form of dead
code elimination.  It notes whether a register has already
been written on the current path through the current VLIW
instruction.  If so, then the previous write is dead and is
deleted.  

Actually there is one exception even to this simple scheme.
If the previous write to the register, also wrote something
else, then it generally cannot be deleted.  For example, if
the previous instruction writing {\tt r3} was {\tt addic
r3,r4,1} and the current instruction is {\tt xor r3,r9,r10},
then the {\tt addic} cannot be deleted because the {\tt CA}
bit would not then be set in the {\tt XER}.  However, leaving
two or more writes to a single GPR can be problematic for some
implementations which logically {\tt OR} the bits of each
operation trying to write to a register.  In this example, the
{\tt addic r3} value and the {\tt xor r3} value would be
incorrectly {\tt OR}'ed together for such implementations.

\subsection{Unification}


As noted in Section~\ref{sect-sched-alg}, the scheduler forms
groups of which are trees of tree instructions (at least in
the most general case).  Thus the same {\it PowerPC} operation
may be represented multiple times in the same group, if it is
past one or more join points in the {\it PowerPC} code.  Such
operations may be aggressively scheduled from multiple paths
so that they are at the same control point in the VLIW code.
For example, consider the following {\it PowerPC} code:

\begin{verbatim}
    nor  r7,r22,r23
    cmpi cr0,r31,0
    beq  L1
    xor  r5,r7,r8
    b    L2
L1:
    and  r9,r7,r10
L2:
    addi r3,r4,1
\end{verbatim}

The final {\tt addi} instruction is common to both paths
through this group, and hence we may get VLIW code such as:

\begin{verbatim}
V1:
         nor  r7,r22,r23
         cmpi cr0,r31,0
         addi r63,r4,1
         addi r62,r4,1
         b    V2

V2:
         beq  VL1
   VL1:  xor  r5,r7,r8
         or   r3,r63,r63
         b V3
         and  r9,r7,r10
         or   r3,r62,r62
         b V4
         

V3:      b    OFFGROUP

V4:      b    OFFGROUP
\end{verbatim}

It is unnecessary for {\tt V1} to have two computations of
{\tt addi rx,r4,r1}.  The unification optimization tracks the
PowerPC address from which each scheduled instruction comes
and notes whether some other path has already computed this
value.  If another path has computed the value, and the
destination register (e.g. {\tt r63}) is unused all along
the path where it is now needed then the operations may
be {\it unified}.  In other words, the VLIW code would be

\begin{verbatim}
V1:
         nor  r7,r22,r23
         cmpi cr0,r31,0
         addi r63,r4,1
         b    V2

V2:
         beq  VL1
   VL1:  xor  r5,r7,r8
         or   r3,r63,r63
         b V3
         and  r9,r7,r10
         or   r3,r63,r63
         b V4
         

V3:      b    OFFGROUP

V4:      b    OFFGROUP
\end{verbatim}

Actually, a variety of other conditions must be met in order
to unify operations.  For more information, look at the {\tt
unify.c} file.  These restrictions sufficiently limit the
number of times when unification may be employed, so as to
yield a relatively small gain from its use.

\subsection{Unrolling}

Because of the limited time available for translation, {\bf
DAISY} does not currently do software pipelining.  It instead
unrolls several times so as to get good overlap among the
original loop iterations.  To do this, {\bf DAISY} checks if
the {\it PowerPC} instruction currently being translated has
been seen along this path previously.  If it has, then this
path contains a loop in the original {\it PowerPC} code.
After passing the loop header $N$ times (where $N$ may
be specified in the {\tt daisy.config} file and is typically
2), {\bf DAISY} serializes.

That is, {\bf DAISY} starts a new group at this loop header
address.  This has the effect of peeling the first $N$
iterations from the loop, and is useful for small iteration
count loops.  This new group formed by {\bf DAISY} gets $M$
copies of the loop body and then loops back to the start of
this group.  $M$ is also specified in {\tt daisy.config} and is
typically 4.

\section{{\bf DAISY} Data Structures}
\label{sect-data-struc}

The two most fundamental data structures used by {\bf DAISY} are in
the file {\tt vliw.h}.  They are the {\bf VLIW} {\tt struct}
defining a VLIW tree instruction and {\bf TIP}, the {\tt struct}
which defines an edge of a VLIW tree instruction.  (For a
description of VLIW tree instructions, see
Section~\ref{sect-tree-ins-mach-model}.)  There are 9 fields
in the {\bf VLIW} {\tt struct}:

\begin{itemize}

  \item {\bf time} indicates how many VLIW instructions distant
	from the start of the group, the current VLIW
	instruction is.  The first VLIW instruction of a group
	has {\bf time} 0.

  \item {\bf group} is a unique identifier indicating the VLIW
	group to which this group to which this instruction
	belongs.  As noted in Section~\ref{sect-sched-alg}, VLIW
	groups are trees of tree instructions.

  \item {\bf group\_entry} is the {\it PowerPC} address of the entry point
	of the {\bf group} to which this VLIW instruction belongs.

  \item {\bf entry} points to the first edge (or {\bf TIP}) in
	the current VLIW instruction.  For example for the VLIW
	instruction in Figure~\ref{fig-vliw-tree} on
	page~\pageref{fig-vliw-tree}, the {\bf entry} edge is
	the segment from {\tt V1} to {\tt bc CR0.eq}.

  \item {\bf xlate} is the address of the simulation code for
	this VLIW instruction.  (This value is not filled in
	until scheduling for a page is complete.)

  \item {\bf visited} is a bit used by the simulation code
	generator and by the VLIW code dumper to know whether
	they have processed this VLIW instruction previously.

  \item {\bf resrc\_cnt} is used to determine the number of each type
	of function unit which have been used by this VLIW
	instruction.  As noted in
	Section~\ref{sect-tree-ins-mach-model}, this version of {\bf
	DAISY} does not distinguish between floating point and integer
	operations.  This is reflected in {\bf resrc\_cnt}, which sees
	only 1 type of ALU.  To change this, the {\bf ALU} {\tt
	\#define} earlier in {\tt vliw.h} must be split into {\tt
	\#define} {\bf ALU\_INT} and {\tt \#define} {\bf ALU\_FLT}.

  \item {\bf cluster} is similar to {\bf resrc\_cnt} but is used
	when {\bf DAISY} is targetting an architecture with clustered
	function units.

  \item {\bf mem\_update\_cnt} is used with the combining
	optimization.

\end{itemize}

The {\bf TIP} {\tt struct} is used extensively by the {\bf DAISY}
scheduler described in Section~\ref{sect-sched-alg}.  Its name
derives from the fact that the last edge or {\it tip} on the
current scheduling path is where the current {\it PowerPC} operation
or the {\it commit} of the current {\it PowerPC} operation's value to
a {\it PowerPC} register is placed.  Some of the more important fields
in the {\bf TIP} {\tt struct} are:

\begin{itemize}

  \item {\bf vliw} is a back pointer to the overall VLIW.  This
	is essential in determining whether a function unit is
	available to add an operation to the current tip.

  \item {\bf op} is a linked list of the operations on the
	current tip.  For example, in Figure~\ref{fig-vliw-tree}
	on page~\pageref{fig-vliw-tree}, the initial {\bf TIP}
	beginning at {\tt V1} would have an {\bf op} field of
	{\tt cmp, add}.

  \item {\bf num\_ops} is the number of operations in the {\bf op}
	list.

  \item {\bf left} and {\bf right} are the false and true paths
	from a conditional jump in a VLIW tree instruction (as
	described in Section~\ref{sect-tree-ins-mach-model}).
	In Figure~\ref{fig-vliw-tree}, {\bf left} would be the
	edge from {\tt bc CR0.eq} to {\tt b V2}, while {\bf
	right} would be the edge from {\tt bc CR0.eq} to {\tt bc
	CR1.lt}.  The terminating branch of an edge, such as
	{\tt bc CR0.eq} is associated with the edge above it.
	Note that unconditional onpage branches entirely
	disappear, with the succeeding ALU operations scheduled
	as though they were in a straight line with the
	predecessor of the unconditional branch.  Leaf branches
	of a VLIW tree instruction have {\bf left} set to {\bf
	0}.  Offpage leaf branches place the address of the
	offpage target in {\bf right}.

  \item {\bf gpr\_vliw}, {\bf fpr\_vliw}, and {\bf ccr\_vliw}
	are used in allocating non-{\it PowerPC} registers for rename
	results.  They are bit vectors, with one bit per
	register to indicate whether the register is free in the
	current tip.  Note that for a register to be free for a
	result, it must not only be free in the current tip, but
	in all tips preceding the current tip until the tip at
	which the result is calculated.  For example, consider
	Figure~\ref{fig-ppc-to-vliw} on
	page~\pageref{fig-ppc-to-vliw}, when scheduling the
	{\it PowerPC} {\tt xor} operation.  The current tip begins at
	{\tt VLIW2}.  The rename register {\tt r63} had to be
	free not only in this tip, but in its predecessor, the
	tip from {\tt bc L1} to {\tt b VLIW2}.

  \item {\bf gpr\_rename}, {\bf fpr\_rename}, and {\bf ccr\_rename}
	are used to determine the register in which a value
	resides at a particular time.  If, as with the {\tt xor}
	result in Figure~\ref{fig-ppc-to-vliw}, an operation is
	renamed, subsequent operations have to know when and
	where to get the rename value and when to get the value
	from the {\it PowerPC} register.  For example the {\tt and}
	which uses the result of the {\tt xor} in
	Figure~\ref{fig-ppc-to-vliw} must know to obtain its
	input from {\tt r63} instead of {\tt r4} in {\tt VLIW2}.

  \item {\bf ca\_rename} and {\bf ov\_rename} are similar to
	{\bf gpr\_rename} above except they indicate whether the
	{\tt CA} and {\tt OV} bits from the {\it PowerPC} {\tt XER}
	register have been renamed in a GPR extender.  The need
	for this is covered in more detail in
	Section~\ref{sect-ppc-idio}.

  \item {\bf stores} and {\bf last\_store} are used for tracking
	stores on the current path.  This is essential for the
	{\it must-alias} optimization described in
	Section~\ref{sect-must-alias}, as well as in knowing
	when a {\tt load-verify} operation is needed on a
	uniprocessor.  (See Section~\ref{sect-load-verif}.)

  \item {\bf avail} indicates the time at which each machine
	resource (not just registers) is available.  This is
	used heavily in the scheduling algorithm described in
	Section~\ref{sect-sched-alg}, and roughly speaking the
	value returned by the {bf get\_earliest\_time()}
	function in {\tt vliw.c} is the maximum {\bf avail} time
	of each of an operation's inputs.

  \item {\bf cluster} is a supplement to {\bf avail}.  In a
	machine with function unit clusters, a result is ready
	at its {\bf avail} time only in its own cluster.

  \item {\bf loophdr} is a bit vector to track which
	instructions have been seen on the current path from the
	entry point of this VLIW group.  If an instruction has
	been seen previously, it means there is a loop.

  \item {\bf gpr\_writer} and {\bf mem\_update\_cnt} are used by the
	combining optimization.

  \item {\bf prob} is the probability that this tip executes
	given that the entry point of this group executed.
	These probabilities are established through a simplified
	Ball-Larus set of heuristics.  These
	probabilities inevitably become small after a few
	conditional branches are encountered on the current
	path, thus making it difficult to know which operations
	to schedule --- a problem alluded to in
	Section~\ref{sect-intro}.

  \item {\bf orig\_ops} counts the number of {\it PowerPC} operations
	which (1) have their result placed in a {\it PowerPC} register
	on this {\bf TIP}, or (2) are a store, or (3) are a
	branch.  This information is used to calculate the
	infinite cache ILP attained, as described in
	Section~\ref{sect-user-view}.

  \item {\bf br\_condbit} specifies which of the (up to 256)
	condition code bits is being tested by the conditional
	branch terminating the {\bf TIP}.  The sense of the
	condition (branch if {\tt TRUE} or branch if {\tt
	FALSE}) is obtained from {\bf br\_ins}, the original
	{\it PowerPC} conditional branch.

\end{itemize}

Another set of important data structures are the opcode formats
in {\tt dis.h}.  The {\bf OPCODE\_BASE} {\tt struct} defines
basic information used in parsing {\it PowerPC} operations such as the
primary and extended opcodes.  Each {\it PowerPC} operation also has a
unique number indicated by the {\bf op\_num} field.  The {\bf
op\_num} values are defined in the file {\tt dis\_tbl.h}.  The
{\bf page} field of {\bf OPCODE\_BASE} refers to the page number
in an old 1990 manual describing the Power (not {\it PowerPC})
architecture.  These numbers are probably of little use at this
point.  However, one of the carryovers of this is that the
mnemonics used in {\tt dis\_tbl.h} and {\tt dis\_tbl.c} are
generally Power, not {\it PowerPC} mnemonics.  This should probably be
changed at some point.

The file {\tt dis\_tbl.c} contains a list of all the Power and
{\it PowerPC} opcodes formated to initialize the {\bf OPCODE1} {\tt
struct} in {\tt dis.h}.  {\bf OPCODE1} has two fields, the {\bf
OPCODE\_BASE} just described and {\bf OPERAND\_IN}.  The {\bf
OPERAND\_IN} {\tt struct} lists the source and destination
operands of each Power and {\it PowerPC} operation.  Operands can be
either explicit in the opcode format, in which case {\tt
dis\_tbl.c} used a field name such as {\tt RT} for them, or
implicit, in which case {\tt dis\_tbl.c} explicitly identifies
them.  For example {\tt CA} is an implicit destination of {\tt
addic} (or {\tt ai} for Power).  The field names such as {\tt
RT} are defined in the file {\tt opcode\_fmt.h}.  Fields are
divided into those such as {\tt RT} which are renameable
(generally corresponding to GPR's, FPR's, and condition code
fields/bits), and those which are not renameable.

Returning to the file {\tt dis.h}, the {\bf OPCODE2} {\tt
struct} is defined as well.  {\bf OPCODE2} is similar to {\bf
OPCODE1}, with both sharing the same {\bf OPCODE\_BASE}.
However, where {\bf OPCODE1} symbolically names some inputs and
outputs, {\bf OPCODE2} explicitly names all inputs and outputs
of a parsed instruction.  For example, {\bf OPCODE1} deals with
the general instruction, e.g. {\tt and RA,RS,RB}, while {\bf
OPCODE2} deals with a specific instance of it, e.g. {\tt and
r3,r4,r5}.

{\it These {\bf OPCODE2} format instructions define not only the
input {\it PowerPC} instructions to {\bf DAISY}, but also the VLIW primitive
operations.} This ties the VLIW architecture very closely (too
closely) to the {\it PowerPC} architecture.  This should be changed at
some point so the VLIW architecture can define its own set of
primitive ALU and memory operations.

\section{Translation Idiosyncrasies}
\label{sect-idio}

\subsection{{\it PowerPC} Translation Idiosyncrasies}
\label{sect-ppc-idio}

{\it PowerPC} has many branch instructions which decrement the
{\tt ctr} and branch depending on whether {\tt ctr} is zero,
possibly in conjunction with some other condition.  Such
branches become serializing, since they both read and set {\tt
ctr}.  In a practical terms, such branches limit parallelism by
requiring that no more than one loop iteration execute per
cycle.  To overcome this problem, it is useful to make {\tt ctr}
one of the non-{\it PowerPC} architected GPR's.  For example
{\bf DAISY} makes the {\it PowerPC} {\tt ctr} synonymous with {\tt r32}.  In
this way the value in {\tt ctr} can be explicitly decremented
with the result renamed (e.g. to {\tt r63}, then committed to
{\tt ctr/r32}.  The renamed value can also be explicitly
compared to 0, and {\tt and}'ed with some other condition if
need be.  In programs with small tight loops, we have observed
significant improvement from these actions.

In addition to branching on the {\tt ctr} value, {\it PowerPC}
branches can of course branch on values of condition register
bits.  It can further branch on the conjunction of a condition
register bit values and whether the {\tt ctr} is 0.  For good,
high frequency performance, such compound branches must be
broken into simple primitives, which ultimately branch based
solely on the value of a single bit.

For similar reasons, {\bf DAISY} places the {\it PowerPC} link
register in {\tt r33}.  The {\tt MQ} register from the Power
architecture is in {\tt r34}.  This should probably be changed at some
point since the {\it PowerPC} architecture does not define the MQ
register.

Although it does not do so in this version, it would be worthwhile for
{\bf DAISY} to maintain a non-{\it PowerPC} architected register such
as {\tt r35} with the value 0.  (This convention could be enforced by
software, i.e. the {\bf DAISY} translator, not by hardware.  Since the
user has no access to the non-{\it PowerPC} registers, this can work.)
Because the {\it PowerPC} load and store instructions, as well as {\tt
addi} follow the convention that if the {\tt RA} field is 0, it means
literal 0, not the contents of {\tt r0}, having a VLIW register
dedicated to 0 makes scheduling much easier.  To see this more
clearly, consider a load instruction with {\tt RA} field {\it not}
equal to 0.  Assume this load is scheduled early with its result
placed in a rename register.  Because of copy propagation, at the time
of this early load the value of the {\tt RA} register may not be in
the same register as indicated in the {\it PowerPC} operation, and may
in fact be in {\tt r0}.  For example, the original {\it PowerPC}
operation may be {\tt lwz r3,8(r4)}.  However, at the time the {\tt
lwz} is scheduled, the value of {\tt r4} may reside in {\tt r0}.
However the instruction {\tt lwz r3,8(r0)} does not mean load from the
address {\tt <r0+8>}, it means literal address {\tt 8}.  To avoid this
problem, the {\bf DAISY} VLIW architecture ought not define {\tt r0} to
mean literal {\tt 0}.  To handle those loads and {\tt addi}'s where
literal {\tt 0} is needed, {\tt r35} could be used.

Yet another problem arises from the {\it PowerPC} contains {\tt
mtcrf} instruction.  The {\tt mtcrf} instruction moves any
combination of 8, 4-bit condition register fields from a GPR to
the condition code register.  Since the VLIW architecture has
more than 8 condition register fields, extending the {\tt mtcrf}
instruction must be handled.  If the VLIW registers are 64 bits
and the VLIW has 16, 4-bit condition register fields, then a
simple extension of {\tt mtcrf} could be done, although the
instruction encoding would likely use more than 32 bits.  Since,
only one field is moved in many cases the {\bf DAISY} software assumes
the architecture will have an additional modified format, {\tt
mtcrf2}.  The {\tt mtcrf2} instruction has 3 operands:
%%
\begin{enumerate}

  \item One 4-bit condition register as the destination for the
	instruction.

  \item A GPR instruction containing the 4-bit source field for
	the move.

  \item An immediate value specifying which 4-bit field in the
	GPR is to be moved to the condition register.
\end{enumerate}

The {\it PowerPC} condition register must be otherwise dealt
with carefully, as it is addressable in 3 ways, (1) as individual
bits for operations like {\tt crnand} and conditional branches,
(2) as 4-bit condition register fields, as set by {\tt cmp}
type instructions and moved by {\tt mtcrf2} type instructions,
and (3) as a full 32-bit entity, as used with the {\tt mfcr}
instruction for example.  Dependences set at one level, must
of course be observed at other levels.  For example a branch
cannot be moved above its compare.

Finally, the {\tt CA, OV}, and {\tt SO} bits of the {\tt XER}
register require special attention in order to attain maximum
parallelism.  The {\tt addic} instruction in particular is
heavily used to increment loop index variables.  Alas, {\tt
addic} not only bumps the value in its destination GPR, it also
sets the carry value in {\tt CA}.  Unless this {\tt CA} value
can be renamed, {\tt addic} instructions must serialize because
of the output dependence between them --- even if the {\tt CA}
value computed is never used --- which is the case for the vast
majority of code.~\footnote{The {\bf DAISY} compiler does not have
sufficient time to do a liveness analysis to determine with
certainty that the value in {\tt CA} is dead.} It would be
better if compilers always used the {\tt addi} instruction instead of
{\tt addi} in such cases.  However, a large body of code
exists using {\tt addic}.  To get around this problem, {\bf DAISY}
places the value of {\tt CA} in an extender bit of the target
GPR for an operation such as {\tt addic} --- if the {\tt addic}
was executed speculatively and its integer result renamed to a
non-{\it PowerPC} architected register.  When the integer result
is committed to its {\it PowerPC} architected register, the {\tt
CA} extender bit is simultaneously committed to the {\tt CA} bit
in the {\it PowerPC} {\tt XER} register.  If an {\tt addic}
instruction is not executed speculatively, the VLIW can place
the carry value directly in the {\tt CA} bit of the {\tt XER}.
(The architecture can tell speculative operations by their
non-{\it PowerPC} architected destination register.)  The
overflow ({\tt OV}) and summary overflow ({\tt SO}) bits are
handled similarly, except that the {\tt OV} extender bit for a
speculative operation is both placed in the {\tt XER} {\tt OV}
bit as well as {\tt or}'ed with the {\tt SO} bit already in the
{\tt XER} register.  

A similar scheme is needed for the bits of the {\tt FPSCR} register,
and the various combinations in these bits are modified by different
floating point instructions.  Several forms of the floating point move
register instruction ({\tt FMR}) are then needed to handle updating
the appropriate combinations of bits in the {\tt FPSCR} from the {\tt
FPSCR} extender bits which must be added to each floating point
register, just as {\tt CA} and {\tt OV} are added to each general
purpose register.

Even the condition register bits need such extenders if the
instructions {\tt fcmpo} and {\tt fcmpu} are allowed to be renamed and
executed out of order.  {\bf DAISY} currently does {\tt fcmpo} and
{\tt fcmpu} in order.

To make {\bf DAISY} emulate the full {\it PowerPC} architecture
including supervisor code, it is important that the machine model be
complete, or at least easily extendable.  For example, dependencies on
mode bits in the MSR should be modelled as well as dependences on
segment registers and BAT's.  WIMG bits must also be modelled for
correct address translation.  There are machine specific registers
such as HID0 on the 604, which also must be modelled.  In general, it
is probably easiest to pick a specific {\it PowerPC} implementation,
and strictly model it.

\subsection{Other Idiosyncrasies}
\label{sect-other-idio}

There are many data structures in {\bf DAISY} which are live only
during the formation of a single group.  It would be useful to have a
special {\tt malloc} for such structures, with all such memory being
automatically free'd after the translation of each group is complete.
{\bf DAISY} has an incomplete and awkward implementation of this
special {\tt malloc}.

Simulation code must be managed in a parallel fashion to the actual
VLIW binary code.  That is when VLIW code for a group is formed, the
simulation code must also be formed (until we have real hardware).
Likewise when a VLIW group is destroyed the simulation code must also
be destroyed.  Code should be written with this in mind so that VLIW
code and simulation code are handled by paired function calls or some
like-minded construct.

There are several shortcomings to the current {\bf DAISY}
implementation which could and should cleaned up.  First, {\bf DAISY}
has no simple, clean way to get a scratch register for use by the
compiler in generating intermediate values or values needed only by
{\bf DAISY} and not the original {\it PowerPC} code, e.g. the
destination of an {\tt LRA} instruction.  Next, {\bf DAISY} has no
simple clean way to delete operations.  This is particularly painful,
when a {\it PowerPC} instruction maps to several VLIW primitives, and
it is not known ahead of time if those primitives will be able to be
scheduled early.  (This happens, for example, on {\it PowerPC} {\tt
loads} going to VLIW {\tt load} -- {\tt load-verify} operations.)  If
all the operations cannot be scheduled early, then they must often all
be deleted.  The current {\bf DAISY} does not make this easy.
Finally, the current {\bf DAISY} assumes that the VLIW primitive
operations will look very much like {\it PowerPC} operations, (but
with more registers).  This problem results from the front and back
ends sharing a common opcode table.  Clearly this problem should be
avoided in new implementations.

\section{Dumping VLIW Instructions}
\label{sect-vliw-dump}

An assembly language dump of the generated VLIW code is
generated in the file {\tt daisy.vliw} if the user specifies the
{\tt -N} flag when running {\bf DAISY} ({\tt daisy}).  The code to
perform this dump is largely in the file {\tt vliw\_dump.c}.
This code traverses each VLIW group --- i.e. a tree of VLIW tree
instructions --- and emits the VLIW instructions.  The most
natural way to traverse a VLIW group is by starting with the
{\bf entry} {\bf TIP} of the first VLIW instruction and
following its {\bf left} and {\tt right} links in depth-first
fashion to each exit of the group.  Unfortunately this
depth-first traversal does not group together the operations in
a single VLIW tree instruction.  Thus the dumper first traverses
the group to find the individual VLIW tree instructions, and
then for each such instruction emits the code.

\begin{figure}
\begin{verbatim}
V1:
T1_27738b58:                        # 0xd01b7244, Cnts Offset: 0
    cax       r1,r2,r3
    bc        12,2,T1_27738e4c      #             SKIP
T1_27738c54:                        #             Cnts Offset: 0
    xor       r63,r5,r6
    b         T1_27738d50
T1_27738e4c:                        # 0xd01b7260, Cnts Offset: 4
    sf        r9,r10,r11
    b         OFFPAGE

V2:
T1_27738d50:                        # 0xd01b7254, Cnts Offset: 8
    rlinm     r12,r1,3,0,28
    or        r4,r63,r63
    and       r8,r63,r7
    bc        4,5,T1_27739044       #             SKIP
T1_27738f48:                        # 0xd01b7258, Cnts Offset: 8
    b         OFFPAGE
T1_27739044:                        # 0xd01b7268, Cnts Offset: c
    cntlz     r11,r63
    b         OFFPAGE
\end{verbatim}
\caption{Example of VLIW dump output with {\tt -N} flag.}
\label{fig-vliw-dump}
\end{figure}

Emitting the code for a VLIW tree instruction subdivides into
emitting the code for each {\bf TIP} of the instruction.  As
noted in Section~\ref{sect-data-struc} the ALU and memory ops
are contained in the {\bf op} field of the tip (in reverse
order).  The tips are glued together using the {\bf left}, {\bf
right} and {\bf br\_condbit} fields of {\bf TIP}.
Figure~\ref{fig-vliw-dump} illustrates the output corresponding
to the two VLIW instructions in Figure~\ref{fig-ppc-to-vliw} on
page~\pageref{fig-ppc-to-vliw}.

There are several points to note about the dump in
Figure~\ref{fig-vliw-dump}:

\begin{itemize}

  \item Power, not {\it PowerPC} mnemonics are used.

  \item VLIW tree instructions begin at labels of the form {\tt
	Vn:}, where {\tt n} is a count that starts at 1 and
	increases for the entire execution of {\bf DAISY}.

  \item {\bf TIP} edges are labelled as {\tt Tn\_12345678},
	where the {\tt n} indicates which invocation of the
	{\bf DAISY} translator produced this tip.  As can be seen from
	Figure~\ref{fig-vliw-dump}, this count begins at 1.  To
	be more precise, this count is the {\bf
	xlate\_entry\_cnt} variable bumped on each invocation of
	the {\bf xlate\_entry()} function discussed in
	Section~\ref{sect-internal-view}.  The {\tt 12345678} is
	the hex address of the {\bf TIP} {\tt struct} for this
	tip, as described in Section~\ref{sect-data-struc}.  The
	{\bf TIP} {\tt struct} memory is reclaimed after each
	page translation, so an address does not uniquely
	identify a tip.  However the $<$ {\it invocation count},
	{\bf TIP} {\it address} $>$ pair do.

  \item Onpage branches and intra-VLIW conditional branches use
	tip labels to indicate their targets.  For example, the
	first {\tt bc} operation in VLIW instruction {\tt V1}
	branches to the last tip in {\tt V1}.  As discussed in
	Section~\ref{sect-tree-ins-mach-model}, this {\tt bc} is
	not really a branch, but a mask to indicate which
	operations in {\tt V1} should execute, {\tt cax, xor} or
	{\tt cax, sf}.  Such conditional branch/mask
	instructions are commented with a {\tt SKIP} annotation
	in the dump as can be seen in
	Figure~\ref{fig-vliw-dump}.  The {\tt b T1\_27738d50} in
	the middle {\bf TIP} of {\tt V1} is actually a branch to
	the next VLIW instruction {\tt V2}.

  \item The hex numbers after the {\tt \#} such as {\tt
	0xd01b7244} in {\tt V1} indicate the address of the
	{\it PowerPC} branch terminating the tip.  In other words, the
	{\tt bc 12,2,T1\_27738e4c} was at address 0xd01b7244 in
	the original {\it PowerPC} program.  Note that the branch terminating
	the middle tip of {\tt V1} does not have an address.  This
	is because this branch does not correspond to any {\it PowerPC}
	branch, but instead to a branch between VLIW instructions.

  \item Each exit tip of a VLIW instruction has a counter
	associated with used for tracking the number of times
	that this tip executes during the execution of the
	translated program.  The {\tt Cnts Offset} values in
	Figure~\ref{fig-vliw-dump} refer to the location of the
	associated exit tip's count value as an offset from the
	start of the counts area.  These {\tt Cnts Offset}
	values can sometimes be useful in debugging as described
	in Section~\ref{sect-debugging}.  {\it Caution:} For
	large programs like {\tt gcc}, the counts area may not
	be large enough to hold a count value for all tips.  In
	such cases the {\tt -I} flag can be specified with {\tt
	daisy} to suppress generation of such counts.  (This also
	speeds simulation at the expense of not collecting as much
	many statistics on the translation.)

\end{itemize}

\section{VLIW Simulation Code}
\label{sect-simul-code}

Most of the simulation code resides in three files, {\tt
simul.c}, {\tt ppc\_map.c}, and {\tt ccr-v2p.c}.  The entry
point for generating simulation code is {\bf
vliw\_to\_ppc\_group()} in {\tt simul.c}.  In much the same
manner as a VLIW assembly language listing is dumped, {\bf
vliw\_to\_ppc\_group()} traverses the VLIW instructions in a
{\it group} and generates {\it PowerPC} code to emulate the action of
the VLIW instructions.  (Section~\ref{sect-vliw-dump} describes
how VLIW instructions are dumped.)

\begin{figure}
\begin{verbatim}
V1_SIM:
    cax     r31,r2,r3                   # cax   r1,r2,r3
    stw     r31,R1_SHADOW_OFFSET(r13)
    lwz     r31,CR0_OFFSET(r13)         # bc    12,2,T1_27738e4c
    mtcrf   0x80,r31
    bc      4,2,LSIM1                   # Inverted branch sense
    liu     r31,LSIM2_HIGH              # Construct target addr
    oril    r31,r31,LSIM2_LOW           # i.e. LSIM2
    mtlr    r31
    blrl                                # Branch to orig bc targ

LSIM1:
    xor     r31,r5,r6                   # xor   r63,r5,r6
    stw     r31,R63_SHADOW_OFFSET(r13)
                                        # VLIW regs: par semantics
    lwz     r1,R1_SHADOW_OFFSET(r13)    # Set r1 at end of V1
    lwz     r31,R63_SHADOW_OFFSET(r13)  # Set r63 at end of V1
    stw     r31,R63_OFFSET(r13)

    liu     r31,V2_SIM_HIGH             # Construct V2 simul addr
    oril    r31,r31,V2_SIM_LOW
    mtlr    r31
    blrl                                # Branch to V2 simul code

LSIM2:                                  # Targ of bc 12,2,T1_27738e4c
    subfc   r31,r10,r11                 # sf    r9,r10,r11
    stw     r31,R9_SHADOW_OFFSET(r13)
                                        # VLIW regs: par semantics
    lwz     r1,R1_SHADOW_OFFSET(r13)    # Set r1 at end of V1
    lwz     r9,R9_SHADOW_OFFSET(r13)    # Set r9 at end of V1

    liu     r27,0xd01b                  # Offpage target address
    oril    r27,r27,0x8180              # is 0xd01b8180 ==> r27
    lil     r26,1                       # Direct Offpage Branch=1

    liu     r31,xlate_entry_raw_HIGH    # Construct xlate_entry_raw
    oril    r31,r31,xlate_entry_raw_LOW # address
    mtlr    r31
    blrl                                # Branch to xlate_entry_raw
\end{verbatim}
\caption{Generation of Simulation Code for a VLIW Tree Instruction.}
\label{fig-simul-code-gen}
\end{figure}

Figure~\ref{fig-simul-code-gen} has simulation code for VLIW
instruction {\tt V1} in Figure~\ref{fig-vliw-dump}.  This simulation
code is generated as follows:

\begin{itemize}

  \item The first ALU/memory operation of {\tt V1} is {\tt cax
	r1,r2,r3}.  Thus the simulation code must perform this
	{\tt cax} (or {\tt add}) operation as well.  Most of the
	VLIW registers are kept at known memory locations.
	However, for efficiency reasons and for ease of
	interfacing to other {\it PowerPC} code, some of the VLIW
	registers are actually kept in {\it PowerPC} registers.  In
	particular, VLIW registers {\tt r1-r12} are kept in real
	{\it PowerPC} registers {\tt r1-r12}.  Likewise floating point
	registers {\tt fp0-??} are kept in real {\it PowerPC}
	registers.  Thus the inputs for the {\tt cax}
	instruction come directly from {\it PowerPC} registers {\tt
	r1} and {\tt r2} as shown in
	Figure~\ref{fig-simul-code-gen}.

	The destination, {\tt r1}, is of course also kept in a
	real {\it PowerPC} {\tt r1}.  However, recall from
	Section~\ref{sect-tree-ins-mach-model}, that VLIW
	instructions use parallel semantics, and hence any uses
	of {\tt r1} in VLIW instruction {\tt V1} should get the
	old value of {\tt r1}, not the new value computed by
	{\tt cax}.  Thus the {\tt cax} result is placed in {\tt
	r31}, and then placed in a shadow version of {\tt r1}.
	At the end of the VLIW instruction, this shadow version
	of {\tt r1} will be copied to the real {\tt r1}, as
	explained below.

	Since only VLIW integer registers {\tt r1-r12} are kept
	in the corresponding {\it PowerPC} register, the other {\it PowerPC}
	integer registers are available for scratch computation.
	In particular note:

  \begin{itemize}

    \item {\tt r31} is used to hold the result of the {\tt cax}.

    \item {\tt r13} is used to access the location of the shadow
	  register for {\tt r1}.  {\tt r13} is used throughout
	  the simulation code to access needed values -- not
	  just registers and their shadows, but simulation
	  counts, hash tables, and other values.  This is
	  analogous to the use of {\tt r2} in AIX code as the
	  {\tt TOC} pointer.

  \end{itemize}

  \item The next operation in {\tt V1} after the {\tt cax} is a
	{\tt bc}.  Note that the {\tt bc} in
	Figure~\ref{fig-vliw-dump} reads condition register bit
	{\tt 2}.  Thus the simulation code must first read bit
	{\tt 2} as well.  The simulator keeps all 64-256 bits of
	the condition register in memory --- none are kept in
	real {\it PowerPC} registers, as with some of the integer and
	floating point registers.  The condition register bits
	are kept as 4-bit fields, with one 4-bit field stored in
	the high order 4 bits of a full 32-bit word.

	Thus the simulation code in
	Figure~\ref{fig-simul-code-gen} first loads the 4-bit
	field ({\tt CR0}) from memory into {\it PowerPC} register {\tt
	r31}.  In order to branch on bit {\tt 2}, the value of
	{\tt CR0} must be placed in the real {\it PowerPC} condition
	register, thus the {\tt mtcrf} instruction which places
	the VLIW {\tt CR0} value in the {\it PowerPC} {\tt CR0} field.

	It then remains to actually branch.  The simulation code
	corresponding to the target of the branch can be
	arbitrarily far away.  Why can the target be arbitrarily
	far away?  Note that there may not be room in the
	currently allocated chunk of memory to place all of the
	simulation code for the current VLIW instruction.  In
	such cases, the simulator places an unconditional branch
	to an arbitrary 32-bit location where the simulation
	code continues.  The code for to do this branching is in
	the {\bf dump()} function in {\tt xlate\_mem.c}.

	Since the target can be arbitrarily far, the sense of
	the conditional branch is inverted, so that the original
	branch target occurs on the fall-through.  In the VLIW
	code of Figure~\ref{fig-vliw-dump}, the operation is
	{\tt bc 12,2} --- branch if bit {\tt 2} is on, whereas
	in the simulation code of
	Figure~\ref{fig-simul-code-gen}, the operation is {\tt
	bc 4,2} --- branch if bit {\tt 2} is off.  Immediately
	following the simulation fall-through are 4
	instructions, culminating with {\tt blr}, to construct
	and branch to the 32-bit address of the simulation code
	for the target, {\tt LSIM2} of the original branch.

  \item {\tt LSIM1} in Figure~\ref{fig-simul-code-gen} has the
	code for the fall-through of the VLIW branch and the
	target of the simulation code branch.  At the
	fall-through of the VLIW code is the operation {\tt xor
	r63,r5,r6}.  Just as with the {\tt cax} operation
	earlier, the {\tt xor} result is computed into {\it PowerPC}
	{\tt r31} and then saved in a shadow location for {\tt
	r63}.

  \item The {\tt xor} operation terminates the ALU/memory
	operations on this path through VLIW instruction {\tt
	V1}.  Thus the values in the shadow registers can be
	copied into the real registers.  The next three
	operations in Figure~\ref{fig-simul-code-gen} do this.
	Since, as noted above, VLIW {\tt r1} is kept in {\it PowerPC}
	{\tt r1}, setting the real register for {\tt r1}
	consists of a single {\tt lwz} operation from the shadow
	location to {\tt r1}.  VLIW {\tt r63} is not kept in a
	{\it PowerPC} register, so its shadow is first loaded into
	{\it PowerPC} {\tt r31}, and then stored into the memory
	location for VLIW {\tt r63}.

  \item Once this is done the simulation code must branch to the
	next VLIW instruction on this path, {\tt V2}.  The full
	32-bit address of the simulation code for {\tt V2} is
	constructed into {\it PowerPC} {\tt r31}, the moved to the
	{\it PowerPC} {\tt LinkReg}.  The simulation code for {\tt v2}
	is then jumped to.

  \item It remains to look at the simulation code for the target
	of the conditional branch in {\tt V1} and the
	fall-through of the simulation code that xbranch.  This
	code is at {\tt LSIM2} in
	Figure~\ref{fig-simul-code-gen}.

  \item The operation starting this edge in the VLIW code in
	Figure~\ref{fig-vliw-dump} is {\tt sf r9,r10,r11}.  This
	{\tt sf} (or {\tt subfc}) is handled just as were the
	{\tt cax} and {\tt xor} operations earlier, with the
	result being stored in the shadow location for {\tt r9}.

  \item With the {\tt subfc} is complete, the ALU/memory
	operations for this path through VLIW {\tt V1} are
	finished.  Just as on the other path, shadow values are
	copied to their real registers.  Since {\tt r1} and {\tt
	r9} were set on this path, these registers are updated.

  \item The culmination of this path differs from the other path
	however, in that the target is not another VLIW instruction,
	but an {\tt OFFPAGE} location.

  \item Branching offpage works as follows in
	Figure~\ref{fig-simul-code-gen}.  {\it PowerPC} register {\tt
	r27} is loaded with the {\it PowerPC} {\tt OFFPAGE} address,
	{\tt 0xD01B8180} in this case.  {\it PowerPC} register {\tt
	r27} is loaded with the branch type ({\tt 1} for {\tt
	OFFPAGE}) of this branch.  Then the address of the function
	{\tt xlate\_entry\_raw} is constructed into {\it PowerPC}
	register {\tt r31}, which is then copied to the {\it PowerPC}
	{\tt LinkReg}.  The final {\tt blr} goes to {\tt
	xlate\_entry\_raw}, whose function was described in
	Section~\ref{sect-internal-view}.
	

\end{itemize}

The generation of the simulation code just described is largely
carried out by functions in the files {\tt simul.c}, {\tt
ppc\_map.c}, and {\tt ccr-v2p.c}, as noted above.  The routines
in {\tt simul.c} generally handle the control flow through the
VLIW instruction, while the routines in {\tt ppc\_map.c} deal
with the ALU and memory operations.  Operations on the condition
register --- {\tt cmp, fcmpo, crnand, mcrf}, etc. are dealt with
in {\tt ccr-v2p.c}.

The simulation code in Figure~\ref{fig-simul-code-gen} is
slightly simplified from the simulation code often generated.
For example, there is no code to increment the number of times
each VLIW exit tip is executed.  The code to do such a bump
follows the style of code in Figure~\ref{fig-simul-code-gen},
loading the address of the count based on an offset from {\tt
r13}, incrementing the count, and storing it back.

There is also no simulation code to call a cache simulator,
trace dumper or other functions written in C.  The simulation
code for such functions is very similar to the code for dealing
with offpage branches.  Parameters such as a load address are
placed in predefined registers.  Then the address of the {\it
raw} cache handler routine is constructed and placed in the
{\it PowerPC} {\tt LinkReg}.  A {\tt blrl} is then used to branch to
the raw cache handler.  Using a {\tt branch-to-LinkReg-and-link}
instruction permits the raw cache handler to return to the
simulator as with ordinary function calls.  The raw cache
handler copies the parameters from predefined registers to AIX
standard registers, such as {\tt r3, r4, ...}, and then branches
to a cache simulator written in C.

Note that the simulator uses the {\it PowerPC} {\tt LinkReg} to
perform indirect branches.  Since this is the case, the {\tt LinkReg}
for the translated program (e.g. {\tt /bin/ls}) must be kept
elsewhere.  As noted in Section~\ref{sect-ppc-idio}, the {\tt LinkReg}
for the translated program is defined to be {\tt r33}, while its {\tt
CtrReg} is defined to be {\tt r32}, with the Power MQ register
occupying (for now) {\tt r34}.

The offset locations from {\tt r13} are accessed both by
assembly language and C routines.  In C, they are accessed via
the {\bf r13\_area} pointer.  In order to keep the offsets
consistent between the C and assembly language routines, the
offsets are generated automatically into files {\tt
resrc\_offset.h} and {\tt r13\_hdr.s}.  The utility that does
this is {\tt gen\_resrc} and it is created from the {\tt
gen\_resrc.c} source file.  {\bf Do not manually alter the
offsets in {\tt resrc\_offset.h}!}

\section{Debugging {\bf DAISY}}
\label{sect-debugging}

Debugging {\bf DAISY} is often tedious and difficult --- at least
debugging problems that occur in the translated code, as opposed
to the {\bf DAISY} translator itself.  For problems like {\tt assert}
failures in {\bf DAISY} itself, your favorite symbolic debugger can be
employed.  For the translated code however, debugging is
generally done at the assembly language level.  This owes to the
fact that {\bf DAISY} takes as input a binary executable (at least in
application mode) and translates it along with any shared
library functions invoked (1) into VLIW code and (2) into
{\it PowerPC} simulation code for that VLIW code.  Comparison of input
and output do not lend themselves to easy identification of
bugs.

The most useful debugging strategy uses a binary search approach.  On
every crosspage and indirect branch, {\bf DAISY} can optionally stop
execution and (1) continue {\bf DAISY}-style execution, (2) report the
address of the next {\it PowerPC} instruction which would have been
executed, or (3) execute native {\it PowerPC} code directly without
being translated into {\bf DAISY} VLIW code.

Binary search can be used to isolate which crosspage or indirect
branch results in the manifestation of a bug.  For example,
suppose {\tt /bin/ls} does not translate correctly, as may be
evidenced by files not being displayed in the same way they are
if {\tt /bin/ls} is invoked directly.  Then {\bf DAISY} can be run
with a specification to shift from execution of the translated
VLIW version to the {\it PowerPC} version after {\bf N} crosspage or
indirect branches.  With {\tt /bin/ls}, an initial value of
$\mbox{\bf N}=1000$ might be reasonable.  If a correct listing
of files results, then the bug must not occur in the first 1000
crosspage or indirect branches.  Hence {\bf N} must be
increased, to say 2000.  If the output is now incorrect, then
the bug must be between crosspage or indirect branch 1000 and
2000.  Following the binary search technique, {\bf N} is now set
to 1500.  This process continues until the precise crosspage or
indirect branch leading to an error is found.  At this point the
translated code for the problem page can be examined in greater
detail with knowledge that the bug must lie here.  Note that
this is true even if there are multiple bugs.  The binary search
technique catches them in sequence, i.e. the first page with a
bug is found (and fixed), then the second, etc.

From a user perspective, binary search is controlled by the {\tt
-L} and {\tt -M} flags to {\tt daisy}.  The number specified
immediately after {\tt -L} (e.g. {\tt -L1500}) specifies the
number of crosspage or indirect branches before execution
switches to native {\it PowerPC} code.  The number after the {\tt -M}
flag specifies what action to take after {\tt -L} crosspage or
indirect branches have been seen.  {\tt -M2} indicates that
control should switch to native {\it PowerPC} code as described, while
{\tt -M1} indicates that {\bf DAISY} should stop and print out the
{\it PowerPC} target address of the current crosspage or indirect
branch.  {\tt -M1} is normally used after {\tt -M2} has isolated
the offending page to find its {\it PowerPC} address.

The {\tt -L} and {\tt -M} flags set the variables {\tt
max\_dyn\_page\_cnt} and {\tt max\_dyn\_pages\_action} which are
referenced in {\bf xlate\_offpage\_indir()} in {\tt simul.c},
which essentially checks if execution should stop or if {\it PowerPC}
code should be executed directly from this point forward.

As an aside, properly switching from execution of translated
VLIW code to native {\it PowerPC} code is a bit tricky, because
{\it PowerPC} registers must be set in a specific order.  The most
difficult is {\tt LinkReg} since {\tt LinkReg} is used to branch
to the native {\it PowerPC} code.  If {\tt LinkReg} is live at this
point, there is a problem.  A trick to overcome this problem can
be found in the file {\tt to\_native.c}.

As another aside, the problem page normally lies in the first
million crosspage or indirect branches.  The worst case thus far
encountered involved a bug translating {\tt gcc} in which the
bug did not manifest itself until more than 26 million crosspage
or indirect branches, with execution of these 26 million taking
almost an hour each.

Alas even when the problem page is identified, there are still
difficulties.  The bug may lie either in the translation from
{\it PowerPC} to VLIW code or in the translation from VLIW code to
simulation code.  Bugs in the {\it PowerPC} to VLIW code mapping are
generally easier to find.  In addition the amount of VLIW code
for the problem page may be large due to code explosion, making
it difficult to spot the bug by inspection.

One approach in this case is to run a native version of the
problem program under a debugger in one window, and the {\bf DAISY}
version in another window.  Since {\bf DAISY} does not reorder
conditional branches, a correspondence between the native
version and {\bf DAISY} version may be kept.  Often after a few
branches, the {\bf DAISY} version will branch one way and the native
version the other, thus further isolating the bug.

Even without comparing branch by branch, the native version can
be useful in isolating which VLIW instructions to examine for
the bug.  Often the amount of time the native version spends on
the problem page (at the problem point) is short.  In such
cases, a trace of the native execution can be created.  For
example, under {\tt dbx}, the {\tt tracei} command can be used
to do this.  This trace can then be matched to a {\tt daisy.vliw}
file (as described in Section~\ref{sect-vliw-dump}) to identify
which VLIW instructions should execute.  By identifying a small
subset of VLIW instructions to examine, this technique further
facilitates finding the bug by inspection.

Sometimes the translated code crashes with an illegal
instruction or similar problem.  In these cases, the VLIW
simulation code near the crash site can be examined under a
debugger, since crashes normally report the state of the
registers including the {\tt IAR}.  The problem is often to find
what portion of the original {\it PowerPC} program was running at the
time of the crash, and also to find which VLIW instruction was
executing (as opposed to the simulation code address reported by
the debugger).

The tip execution counts discussed in
Section~\ref{sect-vliw-dump} can be useful in this regard.  As
noted in Section~\ref{sect-simul-code} these counts are bumped
at each exit of a VLIW instruction.  These bumps occur
explicitly in the simulation code including the reading the old
value from the proper count offset.  These offsets can be
matched to the count offsets reported in the {\tt daisy.vliw}
file (see Figure~\ref{fig-vliw-dump} on
page~\pageref{fig-vliw-dump}) to find the VLIW instruction that
was executing when the program crashed.  Once the VLIW
instruction is found, the {\it PowerPC} address of the original code
can be found by looking at other annotations in the {\tt
daisy.vliw} file, as discussed in Section~\ref{sect-vliw-dump},
and as can be seen in Figure~\ref{fig-vliw-dump}.

Another useful trick in debugging is to look at the contents of
the {\it PowerPC} link register.  As was seen in
Section~\ref{sect-simul-code}, the simulation code for {\bf DAISY}
does many of its branches indirectly through the {\it PowerPC} {\tt
LinkReg} in order to be able to branch to random 32-bit
addresses.  These branches to link register are normally
{\tt blrl}, i.e. they branch to {\tt LinkReg} and then set
{\tt LinkReg} to the subsequent {\it PowerPC} operation.  When the
translated code crashes at some bad address like 0, the contents
of {\tt LinkReg} often indicate how the simulated code got to
the bad address.

An additional caveat about debugging:  Do not set a breakpoint
for the translated code until that code has actually been
generated.  At least under {\tt dbx} this results in illegal
instruction errors.  This restriction means that a breakpoint
must generally be set in the {\bf DAISY} translator itself, with
the breakpoint triggering after the problem code is generated,
but before it is executed.

Another point of interest about debugging {\bf DAISY} involves
segmentation violations.  Segmentation violations can correctly occur
in translated code because {\bf DAISY} schedules loads speculatively.
Not all speculative loads will actually turn out to be needed, and
those that do not may access arbitrary addresses, and consequently
cause segmentation violations.  To overcome this difficulty, {\bf
DAISY} has its own segmentation violation handler (in {\tt shm.c}).
This handler ignores segmentation violations from loads, but
terminates the program on violations from stores.  This approach
occasionally masks problem segmentation violations in the translated
code or in {\bf DAISY} itself.  Note that when debugging under {\tt
dbx}, the {\tt ignore segv} command must be used so that {\tt dbx}
does not catch and complain of segmentation violations that are
actually of the acceptable type described above.  (The use of {\tt
ignore segv} is seems to cause some problems under {\bf AIX 4.1.5},
but seems fine under {\bf AIX 4.3.2}.)

\section{Post-Processing of {\bf DAISY} Output}
\label{sect-post-proc}

\begin{figure}
\begin{center}
\ \psfig{figure=texfig/dvstats.eps,height=7.8in}
\end{center}
\caption{Usage Information for {\tt dvstats}.}
\label{fig-dvstats-info}
\end{figure}

As described in Section~\ref{sect-user-view}, {\bf DAISY} produces a
variety of output files.  Many of these are produced by the
{\tt dvstats} utility, which takes as input three binary files
produced by {\tt daisy}:
%%
\begin{verbatim}
   ls.vliw_perf_ins_cnts
   ls.vliw_perf_ins_info
   ls.vliw_spec_ins_info
\end{verbatim}
%%
if the input to {\tt daisy} was {\tt /bin/ls}.  As with {\tt
daisy}, a usage summary for {\tt dvstats} is produced by
invoking it with no arguments.  An abridged form of this usage
summary is reproduced in
Figure~\ref{fig-dvstats-info}.

\begin{figure}
\noindent\rule{\linewidth}{0.5mm}
\begin{verbatim}
1840306 VLIW      Instructions Executed over 1 run 
 497645 VLIW User Instructions Executed over 1 run 

3332864 Dynamic PPC Ops in Original Program with shared libs

       PPC Path Length
 1.8 = ----------------
       VLIW Path Length

 785864 Dynamic PPC Ops in Original Program -- User only

       PPC Path Length
 1.6 = ----------------
       VLIW Path Length

Specified ALU Operations Histogram:  4662313 ops ( 2.5 ops / ins)
 0:     657355 ( 35.72%   35.72% cumulative)
 1:     244518 ( 13.29%   49.01% cumulative)
 2:     133254 (  7.24%   56.25% cumulative)
 3:     184201 ( 10.01%   66.26% cumulative)
 4:     137968 (  7.50%   73.75% cumulative)
 5:     117211 (  6.37%   80.12% cumulative)
 6:     174626 (  9.49%   89.61% cumulative)
 7:     116383 (  6.32%   95.94% cumulative)
 8:      74790 (  4.06%  100.00% cumulative)
...
\end{verbatim}
\caption{Start of {\tt ls.histo} file}
\label{fig-ls-histo}
\end{figure}

Figure~\ref{fig-ls-histo} shows the first part of the {\tt
ls.histo} file produced by {\tt dvstats}.  As reported in
Section~\ref{sect-user-view}, the {\tt ls.histo} file reports
the infinite cache ILP attained, along with histograms on how
many operations were packed into each VLIW instruction.  The
ILP information is reported for the application as a whole,
and for that part in the original, non-shared library portion
of the application.

\begin{figure}
\begin{verbatim}
/bin/ls
--------------------------------------------------
Invocations of translator                     333
Number of pages visited                       0
Number of user pages visited                  0
Total number of entry points                  1366
Average number of entry points per page       0
# of groups which were created                1366
# of groups which executed   > 10 times       0
# of groups which executed 5 -  9 times       0
# of groups which executed 2 -  4 times       0
# of groups which executed      1 time        0
# of direct offpage    branches executed      189778
# of indirect CTR      branches executed      37457
# of indirect LINKREG  branches executed      70240
# of indirect LINKREG2 branches executed      0
# of TOTAL  crosspage  branches executed      297475
# of hash values with single entry point      1264
Max # of entry points for any hash value      2
Hash value with Max # of entry points         484
Max # of open paths                           170
Total PPC ins (some xlated mult times)        33759
Total VLIW instructions                       1
Total size of VLIW          code              291264
Size of VLIW code from LOAD_VERIFY fails      2420
Size of VLIW code from LOAD_VERIFY recomps    0
Number of              LOAD_VERIFY recomps    0
# Times LOAD-VERIFY instruc failed            10
# Sites LOAD-VERIFY instruc failed            2
Total size of simulation    code              2835264
Total size of original user code              13896
Average size of translated page of code       0
\end{verbatim}
\caption{Example of {\tt daisy.stats} file produced by {\tt daisy}.}
\label{fig-daisy-stats}
\end{figure}

As also described in Section~\ref{sect-user-view}, {\tt daisy} appends
additional statistics to a file {\tt daisy.stats} in the current
directory.  A sample of this information is shown in
Figure~\ref{fig-daisy-stats}.  A note of caution:  programs which
invoke the {\tt fork()} library can be translated by {\bf DAISY}, but
the statistics produced in {\tt daisy.stats} and the files produced by
{\bf dvstats} are not accurate -- only one branch of the fork is seen.
Luckily few benchmark and standard utility programs that we have run
invoke {\tt fork()}.

\section{Caveats}

\begin{itemize}

  \item {\bf DAISY} (executable {\tt daisy}) has been developed on a
	604E machine.  The simulation code generated may cause
	problems on some other variants of the {\it PowerPC}.

  \item As noted in Section~\ref{sect-vliw-dump}, for large
	programs like {\tt gcc}, the counts area may not be
	large enough to hold a count value for all tips.  In
	such cases the {\tt -I} flag can be specified with {\tt
	daisy} to suppress generation of such counts.  (This
	also speeds simulation at the expense of not collecting
	as much many statistics on the translation.)

  \item Whenever {\bf DAISY} generates code, as with the {\bf dump()}
	function in {\tt xlate\_mem.c}, the {\bf flush\_cache()}
	function in {\tt cache.c} should be called to make sure the
	generated code is flushed from the {\it PowerPC} data cache to
	the {\it PowerPC} instruction cache.

  \item In order to use {\tt daisy}, the {\tt datasize} limit must be
	at least 256 Mbytes.  Under {\it csh} or {\it tcsh}, this can
	be accomplished via the command {\tt limit datasize 512000
	kbytes} (which actually sets it to 512 Mbytes).

  \item In order to use {\tt daisy}, it is important to have a large
	amount of paging space:  at least 256 Mbytes, and preferably
	400 Mbytes.  The amount you currently have may be determined
	by typing {\tt lsps -a}.  Contact your system administrator if
	you need to increase your amount.

\end{itemize}

\section{Building the {\bf DAISY} Executables}
\label{sect-build-daisy}

There are four steps to compile the two {\bf DAISY} executables, {\tt
daisy} and {\tt dvstats}:

\begin{enumerate}

  \item Go to the {\tt daisy} directory of this distribution.

  \item Type {\tt make}.

  \item Go to the {\tt dvstats} directory of this distribution.

  \item Type {\tt make}.

\end{enumerate}

\section{Quick Start Running {\bf DAISY}}
\label{sect-quick-start}

9 steps to run {\bf DAISY}:

\begin{enumerate}

  \item Login to a {\it PowerPC}-604 machine, if available.

  \item Make sure the {\tt datasize} limit is set so that {\bf DAISY}
	can run.  {\bf DAISY} requires at least 256 Mbytes.
	Under {\it csh} or {\it tcsh}, this can be accomplished via
	the command {\tt limit datasize 512000 kbytes}, which gives
	512 Mbytes.

  \item Make sure there is sufficient paging space:  at least 256
	Mbytes, and preferably 400 Mbytes.  The amount you currently
	have may be determined by typing {\tt lsps -a}.  Contact your
	system administrator if you need to increase your amount.

  \item Make sure that you have the directory containing {\tt daisy},
	the {\bf DAISY} executable on your path.

  \item Set the environment variable {\it DAISY\_CONFIG} to where
	the file {\tt daisy.config} can be found.  (A sample version
	is in the {\tt daisy} directory of this distribution.)

  \item Set the environment variable {\it CACHE\_CONFIG} to where the
	files {\tt cache.config} and {\tt tlb.config} can be found.
	(Sample versions are in the {\tt daisy} directory of this
	distribution.)

  \item Type {\tt daisy -V432 /bin/ls}.  Within 10-20 seconds, the normal
	directory listing and a new AIX command prompt should appear.
        The -V432 tells {\tt daisy} that it is running under AIX version
	4.3.2.  You should use the proper version of AIX for the machine
	on which you are running.  This number can be determined by
	using the command:

\begin{verbatim}
         lslpp -i | grep adt | grep bos | grep lib | head -1
\end{verbatim}

        {\bf DAISY} has been tested under three versions of AIX:  {\bf
	AIX 3.2.5, AIX 4.1.5}, and {\bf AIX 4.3.2}, although there has
	not been recent testing under {\bf AIX 3.2.5} and performance
	there is uncertain..

  \item The file {\tt daisy.stats} should be in the current directory,
	and have contents similar to those in
	Figure~\ref{fig-daisy-stats} on page~\pageref{fig-daisy-stats}.

  \item Type {\tt dvstats -c ls .}.  This should complete within a
	couple of seconds.  Look at the file {\tt ls.histo}, which
	lists the infinite cache ILP attained by the VLIW machine.
	The {\tt ls.histo} file should be similar to the one in
	Figure~\ref{fig-ls-histo} on page~\pageref{fig-ls-histo}.
	(The {\tt -c} flag suppresses generation of a large {\tt
	ls.cnts} file.)

\end{enumerate}

\vspace*{-0.20in}

\begin{thebibliography}{10}

\vspace*{-0.10in}

\bibitem{Ebcioglu88}
Kemal Ebcio\u{g}lu.
{\it Some Design Ideas for a {VLIW} Architecture for
Sequential-Natured Software}.  In M.~Cosnard et~al., editor, {\em
Parallel Processing}, pages 3--21.  North-Holland, 1988.  (Proceedings
of IFIP WG 10.3 Working Conference on Parallel Processing).

\bibitem{EbciogluAltman96}
Kemal Ebcio\u{g}lu and Erik R. Altman, 
{\it {\bf DAISY:} Dynamic Compilation for 100\% Architectural
Compatibility}.  Research Report RC 20538, IBM T.J. Watson Research
Center, Yorktown Heights, NY, 1996.

\bibitem{EbciogluAltman97}
Kemal Ebcio\u{g}lu and Erik R. Altman,
{\it{\bf DAISY:} Dynamic Compilation for 100\% Architectural
Compatibility}.  In {\em Proc. of the 24th Annual International
Symposium on Computer Architecture}, pages 26--37, Denver, CO, June
1997. ACM.

\bibitem{EbciogluEtAl98}
Kemal Ebcio\u{g}lu, Jason Fritts, Stephen Kosonocky, Michael Gschwind,
Erik R. Altman, Krishna K. Kailas, and Terry Bright,
{\it An eight-issue tree-{VLIW} processor for dynamic binary
translation}.  In {\em Proc. of the 1998 International Conference on
Computer Design (ICCD~'98) -- VLSI in Computers and Processors}, pages
488--495, Austin, TX, October 1998. IEEE Computer Society.

\bibitem{EbciogluEtAl99a} 
Kemal Ebcio\u{g}lu, Erik R. Altman, Sumedh Sathaye, and Michael Gschwind,
{\it Execution-Based Scheduling for VLIW
Architectures}, In {\em Proceedings of Europar'99}, pages 1269--1280,
Toulouse, France, September 1999

\bibitem{EbciogluEtAl99b} 
Kemal Ebcio\u{g}lu, Erik R. Altman, Sumedh Sathaye, and Michael Gschwind,
{\it Optimizations and Oracle Parallelism with Dynamic Translation},
In {\em Proceedings of Micro-32}, Haifa, Israel, November 16-18, 1999.

\bibitem{GschwindEtAl00} 
Michael Gschwind, Kemal Ebcio\u{g}lu, Erik R. Altman, and Sumedh Sathaye
{\it Binary Translation and Architecture Convergence Issues for IBM System/390}
In {\em Proceedings of ICS-2000}, Santa Fe, New Mexico, May 8-10, 2000.

\bibitem{AltmanEbcioglu00}
Erik R. Altman and Kemal Ebcioglu,
{\it Simulation and Debugging of Full System Binary Translation},
Proceedings of PDCS-2000, Track 4 - Modeling and Analysis, August
8-10, 2000, Las Vegas, NV

\bibitem{DAISYWeb}
DAISY website: {\bf www.research.ibm.com/daisy}

\bibitem{VLIWWeb}
IBM VLIW website: {\bf www.research.ibm.com/vliw}

\end{thebibliography}

\end{document}
