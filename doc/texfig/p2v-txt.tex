\documentstyle[11pt]{article}

\setlength{\oddsidemargin}{-0.19in}
%\setlength{\evensidemargin}{-0.315in}
\setlength{\textwidth}{6.8in}
\setlength{\textheight}{9.8in}
\setlength{\columnsep}{2.0pc}
\setlength{\topmargin}{-0.30in}
\setlength{\headheight}{-0.30in}
\setlength{\headsep}{0.0in}
\setlength{\parindent}{1pc}

\renewcommand{\baselinestretch}{0.90}

\pagestyle{empty}

\begin{document}

\noindent\rule{6.79in}{1.00mm}

\begin{small} 
The conversion begins by placing an empty path whose
continuation is instruction 1 ({\tt add r1,r2,r3} as the sole entry
in the {\tt PathList}.  I.e. {\tt PathList} is initially {\tt
\{[(),1,1.0]\}}, where each triple in the {\tt PathList} is of the
form:  \{{\it $<$List of Instructions in Path$>$, $<$Continuation of
Path$>$, $<$Relative Probability of Reaching the End of This Path
Given That Control Has Arrived at the Entry Instruction$>$}\}.  The
{\it current path} is defined to be the highest probability path in
the {\tt PathList}.  Scheduling then proceeds as follows:

\begin{enumerate}

\item An empty VLIW ({\tt VLIW1}) is created and added to the
current path, and {\it PowerPC} instruction 1, {\tt add r1,r2,r3}, is
inserted in it.  The continuation of the current path now becomes
instruction 2. So {\tt PathList} is {\tt \{[(1),2,1.0]\}}.

\item {\it PowerPC} instruction 2, {\tt bc L1} converts {\tt VLIW1}
from a segment to a tree, with the left branch representing the
fall-through path of {\tt bc} and right branch representing its
target, {\tt L1} (instruction 8).  Note that the condition for {\tt
bc} has been computed prior to {\tt VLIW1}, and hence the {\tt bc}
and {\tt add} can be executed in parallel, assuming resource
constraints allow it.  Since instruction 2 is a branch, the current
path is removed from {\tt PathList} and two new paths are created,
one whose continuation is the fall-through instruction 3, and
another whose continuation is the target {\tt L1} (instruction 8).
Both new paths are placed in the {\tt PathList}, making it {\tt
\{[(1,2),3,0.7], [(1,2),8,0.3]\}}.  Assume the branch of instruction
2 is guessed to be taken 30\% of the time.

\item Since the fall-through path is calculated to have a
higher probability it becomes the current path.  {\it PowerPC}
instruction 3, {\tt sli r12,r1,3} depends on the result of
instruction 1, {\tt add r1,r2,r3}.  Hence it must go to a new VLIW.
Hence {\tt VLIW2} is created on the current path, with the
fall-through tip of {\tt VLIW1} pointing to it. The continuation of
the current path is set to instruction 4.  {\tt PathList} becomes
{\tt \{[(1,2,3),4,0.7], [(1,2),8,0.3]\}}.

\item {\it PowerPC} instruction 4, {\tt xor r4,r5,r6} does not depend
on any result yet produced.  Hence it can be executed in {\tt VLIW1}.
However, in order to maintain precise exceptions, we rename the
result to register {\tt r63} (which is not in the {\it PowerPC}
architecture) and copy {\tt r63} to {\tt r4} in {\tt VLIW2}.  So
{\tt PathList} = {\tt \{[(1,2,3,4],5,0.7], [(1,2),8,0.3]\}}.  If an
exception (say an external interrupt) occurred just before executing
{\tt VLIW2}, the emulated {\it PowerPC} machine appears to have
completed instructions 1 and 2, and is at the point immediately
prior to instruction 3.  The results of instruction 4 are still in
{\tt r63} and are not yet committed to an architected register, at
the point of the interrupt.
%% 
%% If we had allowed instruction 4 to commit its result to an
%% architected register, then it could be in general impossible to
%% identify the precise point in the {\it PowerPC} program we are in,
%% in case of an interrupt before {\tt VLIW2}. Also note that if an
%% operation executed out of order (i.e. whose target is a
%% non-architected register) causes an exception, the exception does
%% not occur, only the exception tag of the target register {\tt r63}
%% is set (each non-architected register has an extra exception tag
%% bit). If the first branch indeed is not taken and the program
%% executes the commit operation {\tt lr r4=r63} in {\tt VLIW2}, the
%% exception then occurs, and the precise {\it PowerPC} instruction
%% responsible for the exception is communicated to the base
%% architecture operating system.

\item {\it PowerPC} instruction 5, {\tt and r8,r4,r7} depends on the
result of the {\tt xor}.  Because of our aggressive scheduling this
result can be used in {\tt VLIW2} by noting that the desired value
of {\tt r4} is in {\tt r63}, yielding {\tt and r8,r63,r7}.  The
continuation of the current path is set to 6.  {\tt PathList}
becomes {\tt \{[(1,2,3,4,5),6,0.7], [(1,2),8,0.3]\}}.

\item {\it PowerPC} instruction 6, {\tt bc L2} has no data
dependences and hence can be scheduled in {\tt VLIW2} in a manner
analogous to {\tt bc L1} being scheduled in {\tt VLIW1}.  But we do
not schedule branches earlier than the last VLIW on a path, in order
to maintain precise interrupts.  The current path is replaced by
two paths: one continuing with the fall-through instruction 7, and
one continuing with the target {\tt L2} (instruction 10). Assume this
second branch is also guessed to be taken with 30\% probability.
{\tt Pathlist} becomes {\tt \{[(1,2,3,4,5,6],7,0.49(0.7$\times$0.7)],
[(1,2),8,0.3], [(1,2,3,4,5,6),10,0.21(0.7$\times$0.3)]\}}.

\item Of the 3 paths, the fall-through path of instruction 6, is now
most likely, so it becomes the current path.  Its continuation is
{\it PowerPC} instruction 7, {\tt b OFFPAGE}.  It is placed on the
left tip of {\tt VLIW2} since branches are scheduled in order.
Since this branch has no onpage continuations, this path is removed
from {\tt PathList}, and the next most probable path becomes the new
current path.  {\tt Pathlist} becomes {\tt \{[(1,2),8,0.3],
[(1,2,3,4,5,6),10,0.21]\}}.

\item The {\it PowerPC} instruction 8 {\tt sub r9,r10,r11}, the {\tt
L1} target, is the continuation of the current, highest probability
path.  This target continues from the right tip of {\tt VLIW1},
since that is the location of the branch that inserted it in {\tt
PathList}.  This {\tt sub} instruction has no data dependences with
earlier instructions, and hence can be scheduled on the right tip of
{\tt VLIW1}.  The continuation of the current path becomes 9, and
{\tt PathList} becomes {\tt \{[(1,2,8),9,0.3],
[(1,2,3,4,5,6),10,0.21]\}}.


\item {\it PowerPC} instruction 9, {\tt b OFFPAGE} is next on the
current path.  It is handled just like instruction 7, and hence a
{\tt b OFFPAGE} is placed on the right tip of {\tt VLIW1}.  This
path is then removed from {\tt PathList}, which now becomes {\tt
\{[(1,2,3,4,5,6),10,0.21]\}}.

\item The only open path remaining in the list is the one that
continues with {\it PowerPC} instruction 10 {\tt cntlz r11,r4}, the
{\tt L2} target from {\tt VLIW2}.  It is dependent on the result of
instruction {\tt 4}, {\tt xor r4,r5,r6}.  As noted, this value of
{\tt r4} is available in {\tt VLIW2} itself in {\tt r63}.  Hence
instruction 10 can be scheduled on the right tip of {\tt VLIW2}.
The continuation of the current path becomes 11, and {\tt PathList}
becomes {\tt \{[(1,2,3,4,5,6,10),11,0.21]\}}.

\item {\it PowerPC} instruction 11, {\tt b OFFPAGE} is next on this
path.  It is handled just like instructions 7 and 9, and hence a
{\tt b OFFPAGE} is placed on the right tip of {\tt VLIW2}.  This
path is then removed from the list.  As there are no more entries in
the PathList, and no more entries to process, the algorithm
terminates.  The translated code is ready for execution beginning at
{\tt VLIW1}.

\end{enumerate}
\end{small}

\vspace*{-0.15in}

\noindent\rule{6.79in}{1.00mm}

\end{document}
